\subsection*{\hypertarget{monster}{Monsters}}
"With each passing day, the world finds new and exciting ways to kill a man."\\
\indent -- Balthier
\vspace*{0.3cm}
\begin{center} \includegraphics[width=\columnwidth]{./art/images/ff6.png} \end{center}
\vspace*{0.3cm}
\addcontentsline{toc}{subsection}{Monsters}%
Monsters are wild beings similar to animals, that inhabit uncivilized places in the world.
They usually have a natural habitat where they try to survive, so on contact with the party they will feel threatened and attack.
Different types of monster will generally work together against the party, though this might not always be the case.
The party might also come across more powerful and intelligent monsters with accordingly more complex goals.
Monsters are often part or cause of ongoing conflicts, so the party will be confronted with them fairly often.

\subsubsection*{Creating Monsters}
The game world that you create will usually feature various monsters, that the party will have to face during their adventure. 
You can use the following guidelines to create your own monsters.

\begin{description}[leftmargin=*]

\item[\color{accent} Context:]
When creating a monster, it is important to consider under what circumstances they will face the party.
Due to the combat rules, the side with more participants is at a great advantage, as they can take more actions per round.
Thus, powerful but outnumbered enemies will often have to make up with significantly stronger attributes and abilities.
Also, make sure to adjust the difficulty of monsters to consider your group's experience with the game.

\item[\color{accent} Attributes:]
The attributes of characters after Level 1 are distributed as follows, which you can use as a rule-of-thumb:
for every Level up a character gains 5 points worth of attributes, where a point equals either 5~HP/MP or 1~STR/DEF/MAG/RES.
\pagebreak
AGI should generally stay between 1 and 5, but monsters with low AGI should make up in other aspects.  
Unlike characters, monsters can also significantly vary in size, which can have a great effect on the battle.
Although monsters do not use traditional weapons and armor, they have equivalent parts integrated into their bodies, that follow the same rules.

\item[\color{accent} Abilities:]
Monsters can also use Magic and Techs, as well Passive and Reaction abilities.
However, aim to keep the number of monster abilities to a minimum, to allow for quick decisions during combat.
Nevertheless, feel free to give monsters access to unique and exotic abilities to make them more interesting. 

\item[\color{accent} Resiliences \& Weaknesses:]
You can add more strategic depth to a monster by utilizing resiliences and weaknesses to specific damage types.
Usually lower Level monsters tend to have more weaknesses, while stronger monster are often resilient against damage types.
Unlike characters, monsters may also be inherently immune to various status effects and damage types.
During combat, you can give subtle hints to the players about the resiliences and weaknesses of an enemy when narrating the combat actions and their effects.

\item[\color{accent} Humanoids:]
If you want to create a humanoid enemy, follow the character rules in the \hyperlink{char}{Characters section}.
Depending on the importance of the enemy, you can omit details that you feel are unnecessary.	
Only for major antagonists, it is worthwhile to fill out an entire character sheet.
Also consider that humanoid enemies can make use of and upgrade equipment and items just like the player characters.	
	
\end{description}

\subsubsection*{Examples}
In the following, some examples of monsters are given that might be encountered in your world.
The Level of a monster vaguely indicates the Level the party should be to fight it.
Furthermore, monsters drop Gil upon defeat, which you may substitute with equipment, items or materials of similar value.
These rewards are evenly divided among the adventurers after each successful battle.
Monsters are classified in size as Medium~(\textbf{M}) if they take up roughly 1u in diameter, as Large~(\textbf{L}) if they take up more than 2u and as Small~(\textbf{S}) if they take up less than 0.5u when viewed from above.
All monsters with a purple box instead of a red one are potentially friendly towards the party and will not attack unless provoked.
You can use the monsters in this chapter as given, but you are also encouraged to make changes to them or use them as examples to create your own.
Therefore, the next page is made up of templates that you can use the create your own monsters. 

\pagebreak

\friendly{\phantom{y}}{\hspace{0.3cm}\phantom{k}}{}
{
	HP: & \hfill  & MP: & \hfill  \\
	STR: & \hfill  & DEF: & \hfill  \\
	MAG: & \hfill  & RES: & \hfill  \\
	AGI: & \hfill  & Size: & \hfill \\
}
{
	\textbf{Weapon}: \\
	\textbf{Weak}:  \\
	\textbf{Resilient}: \\
	\textbf{Immune}: \\
	\textbf{Drops:}   
	\vspace{0.1cm} 
	\hrule 
	\vspace{3cm} 
	\hrule 
	\vspace{3cm} 
}
\vfill
\monster{\phantom{y}}{\hspace{0.3cm}\phantom{k}}{}
{
	HP: & \hfill  & MP: & \hfill  \\
	STR: & \hfill  & DEF: & \hfill  \\
	MAG: & \hfill  & RES: & \hfill  \\
	AGI: & \hfill  & Size: & \hfill \\
}
{
	\textbf{Weapon}: \\
	\textbf{Weak}:  \\
	\textbf{Resilient}: \\
	\textbf{Immune}: \\
	\textbf{Drops:}   
	\vspace{0.1cm} 
	\hrule 
	\vspace{3cm} 
	\hrule 
	\vspace{3cm} 
	\hrule 
	\vspace{3cm} 
}

\pagebreak

\monster{\phantom{y}}{\hspace{0.3cm}\phantom{k}}{}
{
	HP: & \hfill  & MP: & \hfill  \\
	STR: & \hfill  & DEF: & \hfill  \\
	MAG: & \hfill  & RES: & \hfill  \\
	AGI: & \hfill  & Size: & \hfill \\
}
{
	\textbf{Weapon}: \\
	\textbf{Weak}:  \\
	\textbf{Resilient}: \\
	\textbf{Immune}: \\
	\textbf{Drops:}   
	\vspace{0.1cm} 
	\hrule 
	\vspace{3.35cm} 
 	\hrule 
	\vspace{3.35cm} 
	\hrule 
	\vspace{3.35cm} 
 	\hrule 
	\vspace{3.35cm} 
	\hrule 
	\vspace{3.35cm} 
	\hrule 
	\vspace{3.35cm} 
}

\clearpage

%%%%%%%%%%%%%%L1%%%%%%%%%%%%%%
\monster{Esqueleto}{1}{\includegraphics[width=0.14\textwidth]{./art/monsters/skeleton.png}}
{
 PV: & \hfill 12  & PM: & \hfill 0\\
 FUE: & \hfill 2 & DEF: & \hfill 1 \\
 MAG: & \hfill 0 & RES: & \hfill 0 \\
 AGI: & \hfill 2 & Tamaño: & \hfill M \\   
}
{
 \textbf{Espada}: 1d de daño \hfill \textbf{Botín:} 100 Gil \\
 \textbf{Debilidad}:\fire \holy \mpassive{No Muerto}{Sufres permanentemente el estado \hyperlink{status}{Zombi}.}
 %\vspace{0.1cm} \hrule \vspace{0.1cm} 
 %\emph{"¡Je! ¡Estás más alto, pero no eres más que huesos! ¿Estás comiendo bien, muchacho?" -- Jecht}
}


 
\friendly{Mandrágora}{1}{\includegraphics[width=0.11\textwidth]{./art/monsters/mandragora.png}}
{
 PV: & \hfill 10 & PM: & \hfill 16\\
 FUE: & \hfill 2 & DEF: & \hfill 1 \\
 MAG: & \hfill 0 & RES: & \hfill 0 \\
 AGI: & \hfill 2 & Tamaño: & \hfill P\\
}
{
 \textbf{Cabezazo}: 1d de daño \hfill \textbf{Botín:} 100 Gil \\
 \textbf{Debilidad}:\fire \mspell{Morfeo}{8}{1t}{Único}{3u}{El objetivo hace una tirada con DC 8. Si falla, queda \hyperlink{status}{Dormido} por 3 turnos.}{\sleep} %\vspace{0.1cm} \hrule \vspace{0.1cm} 
 %\emph{"No me interesa." -- Cloud} 
}

 
\monster{Tarantula}{1}{\includegraphics[width=0.2\textwidth]{./art/monsters/tarantula.png}}
{
	HP: & \hfill 8 & MP: & \hfill 8\\
	STR: & \hfill 1 & DEF: & \hfill 0 \\
	MAG: & \hfill 0 & RES: & \hfill 0 \\
	AGI: & \hfill 3 & Size: & \hfill S\\
}
{
	\textbf{Bite}: 1d DMG \hfill \textbf{Drops:} 100 Gil \\
	\textbf{Weak}:\fire
	
	\mtech{Web}{4}{1r}{Single}{3u}{The target makes a DC 8 check and suffers \hyperlink{status}{Immobile} for 1 round upon failure.}{\immobile}
	%\vspace{0.1cm} \hrule \vspace{0.1cm} 
	%\emph{"Hi there, creepy crawly." -- Paine}		
}	 
\monster{Goblin}{1}{\includegraphics[width=0.15\textwidth]{./art/monsters/goblin.png}}
{
	HP: & \hfill 10 & MP: & \hfill 0\\
	STR: & \hfill 1 & DEF: & \hfill 1 \\
	MAG: & \hfill 0 & RES: & \hfill 0 \\
	AGI: & \hfill 3 & Size: & \hfill M\\
}
{
	\textbf{Knife}: 1d DMG \hfill \textbf{Drops:} 150 Gil  
	%\vspace{0.1cm} \hrule \vspace{0.1cm} 
	%\emph{"We just be cannonfodder. Never 'ave the chance to show what we're really made of..." -- Goblin}
}

%%%%%%%%%%%%%%L2%%%%%%%%%%%%%%
\monster{Sahagin}{2}{\includegraphics[width=0.12\textwidth]{./art/monsters/sahagin.png}}
{
		HP: & \hfill 14 & MP: & \hfill 24\\
STR: & \hfill 2 & DEF: & \hfill 1 \\
MAG: & \hfill 2 & RES: & \hfill 1 \\
AGI: & \hfill 3 & Size: & \hfill M\\
}
{
	\textbf{Spear}: 1d DMG  \hfill \textbf{Drops:} 150 Gil  
	
	\textbf{Resilient}:\water \hfill \textbf{Weak}:\lightning 	
	\mspell{Water}{8}{1r}{Single}{4u}{
		You deal 3d \hyperlink{type}{water} damage to the target.
	}{\water}
	%\vspace{0.1cm} \hrule \vspace{0.1cm} 
	%\emph{"Something's fishy about this place, and it ain't cod..." -- Bartz}		
}
 
\monster{Ghoul}{2}{\includegraphics[width=0.14\textwidth]{./art/monsters/ghoul.png}}
{
	HP: & \hfill 17 & MP: & \hfill 12\\
	STR: & \hfill 3 & DEF: & \hfill 1 \\
	MAG: & \hfill 1 & RES: & \hfill 2 \\
	AGI: & \hfill 2 & Size: & \hfill M\\
}
{
	\textbf{Claw}: 1d DMG \hfill \textbf{Drops:} 150 Gil \\
	\textbf{Weak}:\fire \holy \hfill \textbf{Resilient}:\ice \\
	\textbf{Immune}:\poison  \hfill 
	
	\mtech{Bite}{3}{0r}{Single}{1u}{
		The target takes 2d damage and makes a DC 8 check.
		Upon failure, he suffers \hyperlink{status}{Zombie} for 1 hour.
	}{\zombie}		
	\mpassive{Undead}{You permamently suffer the \hyperlink{status}{Zombie} status.}
	%\vspace{0.1cm} \hrule \vspace{0.1cm} 
	%\emph{"Being dead has its advantages." -- Auron}
}

 
\monster{Cocatriz}{2}{\includegraphics[width=0.15\textwidth]{./art/monsters/cockatrice.png}}
{
 PV: & \hfill 13 & PM: & \hfill 16\\
 FUE: & \hfill 2 & DEF: & \hfill 1 \\
 MAG: & \hfill 0 & RES: & \hfill 2 \\
 AGI: & \hfill 3 & Tamaño: & \hfill M\\
}
{
 \textbf{Pico}: 1d de daño \hfill \textbf{Botín:} 150 Gil \\
 \textbf{Debilidad}:\lightning \mspell{Ciego}{8}{1t}{Único}{3u}{
 El objetivo hace una tirada con DC 8. Si falla, sufre \hyperlink{status}{Ceguera} por 3 turnos. }{\blind}
	%\vspace{0.1cm} \hrule \vspace{0.1cm} 
 %\emph{"¿Desde cuándo los heroes necesitan planes?" -- Snow} 
} 
\monster{Coeurl}{2}{\includegraphics[width=0.17\textwidth]{./art/monsters/coeurl.png}}
{
	HP: & \hfill 16 & MP: & \hfill 15\\
	STR: & \hfill 2 & DEF: & \hfill 2 \\
	MAG: & \hfill 1 & RES: & \hfill 3 \\
	AGI: & \hfill 3 & Size: & \hfill M\\
}
{
	\textbf{Claw}: 1d DMG \hfill 	\textbf{Drops:} 200 Gil   
	
	\mtech{Blaster}{5}{1r}{Single}{5u}{The target makes a DC 8 check and suffers \hyperlink{status}{Immobile} for 3 rounds upon failure.}{\immobile}	
	\vspace{0.1cm} \hrule \vspace{0.1cm} 
	"Remember what curiosity killed, just a friendly word of advice!" -- Balthier
}
 

%%%%%%%%%%%%%%L3%%%%%%%%%%%%%%
\ofsubsection{Chocobo}
%

\ofquote{"There's no wrong way to love a chocobo."\\}{Noctis}\\\\
%
\includegraphics[width=0.95\columnwidth]{./art/chocobo/chocobo.jpg}
%
\\\\
%
%
\accf{Chocobos} are large, flightless avian creatures with yellow feathers and a long neck.
They are very intelligent and even understand humanoid languages to some degree.
Therefore, Chocobos are often domesticated and used as mounts, making them comparable to horses and renting out Chocobos is a lucrative business for farmers.
Although prices may fluctuate, the party can usually rent a Chocobo for about 10G per day.
In rare cases, farmers also sell Chocobo at extremely expensive prices, starting at around 3000G. 
Alternatively, the party can try to catch Chocobos that roam in the wild, they can usually be found in forests or wide grasslands.
Such Chocobos generally consider them to be hostile by default and engage in combat when feeling threatened.
When taking any damage, a wild Chocobo performs a DC~7 check and upon failure it becomes scared and flees as quickly as possible.
A character can gain its trust by using their action to feed it a Chocobo's favorite food, the \accf{Gysahl Greens}.
In this case, the player performs a check with a DC of 6 + the Chocobo's Level and if successful, it will join the party and follow his or her command from now on.
%
\ofpar
%
%\ofquote{"Fat Chocobo? You're rude! Here it's the bird of gods!"}{Dwarf}\\\\
%
As most avian creatures, Chocobos lay eggs from which their babies hatch.
However, they grow surprisingly quickly: an egg hatches a few weeks after it is laid and after another month, 
most Chocobos are already as large as their owner.  
They are usually bred in stables, where they can be kept in a warm and safe environment.
Chocobos can be of different types, which is determined by the color of their feathers.
The most common one is the yellow Chocobo, other types are rather rare compared to it.
A Chocobo's type depends on its parents and the following table shows the outcome of different pairings.
In all cases that are not listed, a Chocobo has its parents' type if they are both the same and it is yellow otherwise.\\\\
%
\oftable{p{0.37\columnwidth} p{0.37\columnwidth} p{0.3\columnwidth}}
{\accf{Parent 1} & \accf{Parent 2} & \accf{Child}} {
Yellow 	& Blue   & Green \\
Yellow 	& Red    & Green \\
Blue 	& Green  & Red \\
Red 	& Green  & Blue \\
Blue 	& Blue   & White \\
Red 	& Red    & Black \\
Black 	& White  & Gold\\
}
%
\vfill
%
This knowledge is available at many experienced Chocobo breeders or in books about the topic.
The party can try to breed some of the rare types, which often come with special abilities.
Details about the different Chocobo types are shown at the end of this subsection.
In some cases a newly born Chocobo's type might not adhere to the table above.
Whenever a new Chocobo is born, make a DC 11 check and if you succeed, its type is instead determined as follows:
roll 2d, the Chocobo is white on 2-3, blue on 4-5, yellow on 6-8, red on 9-10, black on 11-12.
%
%\ofpar
%
Raising a Chocobo is not a simple task, as they require a lot of care and attention.
In return, a Chocobo can the help the party in various ways through their unique capabilities, which improve throughout the adventure.
As such, the current experience of a Chocobo is tracked through its Level, the same way as for player characters. 
However, Level ups are performed slightly differently for Chocobos.
Firstly, a Chocobo can only learn a pre-determined set of abilities depending on its type.
Secondly, the attribute increases at Level up are also handled differently for Chocobos:
their maximum HP and MP increase both by 5 at each Level up. 
In addition, its owner can spend an additional 3 points to further improve the Chocobo's attributes as desired.
The table below shows how many points need to be spent for different attribute bonuses.
A final noteworthy difference compared to player characters is that Chocobos posses the additional \accf{Stamina~(STA)} attribute, which determines their affinity for long distance travel.
%
\ofpar
%
\oftable{p{0.5\columnwidth} p{0.3\columnwidth}} 
{\accf{Attribute Bonus} & \accf{Required Points}} {
  Max. HP +5 	& 1 \ofrow
  Max. MP +5 	& 1 \ofrow
  STR +1 		& 1 \ofrow
  DEF +1 		& 1 \ofrow
  MAG +1 		& 1 \ofrow
  RES +1 		& 1 \ofrow
  STA +1 		& 2 \ofrow
  DMG +1d 		& 3 \ofrow
  AGI +1 		& 3 \ofrow
}
%
\clearpage
%
%\ofquote{"Man... Chocobo, we just can’t get a break, can we?"\\}{Sazh}\\\\
\ofquote{"My hair does NOT look like a Chocobo's butt!"\\}{Prompto}\\\\
%
Characters can ride Chocobos for a fast and comfortable travel experience.
Riding domesticated ones is simple, but a more experienced rider may come out ahead in sticky situations.
They can carry a reasonable amount of weight without being affected.
Nevertheless, Chocobos get tired after too much uninterrupted travel time.
A Chocobo can walk an amount of hours equal to its Stamina attribute before it needs a break.
This time is halved, if the carried total weight significantly exceeds that of two average humans.
Even though Chocobos usually follow their owner's orders, they might refuse to keep going whenever they are particularly scared or caught by surprise.
Chocobos can also be very capable combatants and thus crucial additions to the party line-up.
They can fight alongside the party, in which case they are treated as any other allied combat participant.
A Chocobo is controlled by the player whose character is its owner and it obeys their commands.
Alternatively, characters can also decide to stay mounted on their Chocobo during combat.
If they do so, the Chocobo and its owner always take their turn together, where only the Chocobo handles the movement. 
Whenever the rider Attacks a small or medium sized enemy while mounted, the target has Disadvantage on the evasion check.
However, when the Chocobo suffers damage while carrying a rider, it has to make a check with a DC of 12 minus its STR attribute. 
If it fails this check, the rider is thrown off and suffers Immobile for 1 round.
%
\vfill
%
\ofmonster{Yellow Chocobo}{1}{\includegraphics[width=0.23\columnwidth]{./art/chocobo/chocyellow.jpg}}
{
	HP: & \hfill 19 & MP: & \hfill 17\\
	STR: & \hfill 1 & DEF: & \hfill 0 \\
	MAG: & \hfill 1 & RES: & \hfill 0 \\
	AGI: & \hfill 2 & STA: & \hfill 2 \\
}
{\accf{Beak}: 1d DMG}
{
	\mspell{Cure (Level 1)}{4}{0r}{Single}{3u}{The target regains 2d HP.}{\accf{Level 1}}		
	\mspell{Esuna (Level 3)}{6}{0r}{Single}{5u}{Remove all Status Effects except KO.}{\accf{Level 3}}	
	\mtech{Enrage (Level 6)}{10}{0r}{Single}{5u}{The target performs a DC 8 check and upon failure he has to move towards you on his next turn and if possible perform an Attack on you.}{\accf{Level 6}}	
	\mtech{Fat Chocobo (Level 9)}{16}{1r}{1u}{5u}{Everyone in the target area suffer 6d damage and Immobile for 1 round.}{\accf{Level 9}\immobile}	
	\mpassive{Choco Glide}{You can glide down slowly from heights up to 30u without taking any damage.}
}
%
\newpage
%
\ofquote{"Ya know, all I want to do is ride on a chocobo. Faster than the wind!"}{Clasko}\vfill
%
\ofmonster{Red Chocobo}{1}{\includegraphics[width=0.23\columnwidth]{./art/chocobo/chocred.jpg}}
{
	HP: & \hfill 21 & MP: & \hfill 13\\
	STR: & \hfill 2 & DEF: & \hfill 1 \\
	MAG: & \hfill 0 & RES: & \hfill 0 \\
	AGI: & \hfill 2 & STA: & \hfill 2 \\
}
{\accf{Beak}: 1d DMG \hfill \accf{Resilient:}\fire}
{	
	\mtech{Choco Kick (Level 1)}{4}{0r}{Single}{Weapon}{The target suffers 2d damage and is knocked back by 1u.}{\accf{Level 3}}
	\mtech{Choco Dash (Level 3)}{7}{0r}{5u (line)}{Self}{You dash in a line of up to 5u dealing 3d damage to everyone in the target area and knocking them to the side by 1u.}{\accf{Level 6}}
	\mtech{Choco Blaze (Level 6)}{14}{0r}{3u}{Self}{Everyone in the target area except you suffers 5d fire damage.}{\accf{Level 9}\fire}
	\mreaction{Choco Counter}{Whenever you are hit by an Attack, immediately make an Attack on the perpetrator.}
	\mpassive{Choco Jump}{You can perform a powerful high jump to cover a distance of up to 10u vertically.}		
}
%
\vfill
%
\ofmonster{Blue Chocobo}{1}{\includegraphics[width=0.23\columnwidth]{./art/chocobo/chocblue.jpg}}
{
	HP: & \hfill 15 & MP: & \hfill 25\\
	STR: & \hfill 0 & DEF: & \hfill 0 \\
	MAG: & \hfill 2 & RES: & \hfill 1 \\
	AGI: & \hfill 2 & STA: & \hfill 2 \\
}
{\accf{Beak}: 1d DMG \hfill \accf{Resilient:}\water}
{	
	\mspell{Water (Level 1)}{6}{0r}{Single}{4u}{You deal 2d water damage to the target.}{\accf{Level 1}\water}	
	\mspell{Accumulate (Level 3)}{3}{0r}{Single}{5u}{The target gains EnMAG for 3 rounds.}{\accf{Level 3}\enmag}	
	\mspell{Waterga (Level 6)}{14}{1r}{Single}{6u}{You deal 6d water damage to the target.}{\accf{Level 6}\water}	
	\mtech{Supersonic Wave (Level 9)}{18}{0r}{3u (front)}{Self}{All enemies in the target area suffer 4d damage and make a DC~8 check. Upon failure they suffer Silence for 3 rounds.}{\accf{Level 9}\silence}
	\mpassive{Choco Swim}{You can swim slowly through any river or sea without excessive current.}		
}
%
\clearpage
%
\ofmonster
{Green Chocobo}{1}{\includegraphics[width=0.23\columnwidth]{./art/chocobo/chocgreen.jpg}}
{
	HP: & \hfill 16 & MP: & \hfill 21\\
	STR: & \hfill 0 & DEF: & \hfill 1 \\
	MAG: & \hfill 1 & RES: & \hfill 1 \\
	AGI: & \hfill 2 & STA: & \hfill 2 \\
}
{\accf{Beak}: 1d DMG \hfill \accf{Immune:}\poison\blind\sleep}
{
	\mspell{Protect (Level 1)}{5}{0r}{Single}{5u}{The target gains EnDEF for 3 rounds.}{\accf{Level 1}\enndef}
	\mspell{Regen (Level 3)}{6}{0r}{Single}{5u}{The target gains regen for 3 rounds.}{\accf{Level 3}}	
	\mspell{Reflect (Level 6)}{10}{0r}{Single}{3u}{The target gains a shield that reflects the next spell that targets them back to its caster.}{\accf{Level 6}}	
	\mspell{Full-Life (Level 9)}{24}{1r}{Single}{3u}{Remove KO status from the target and fully restore his HP.}{\accf{Level 9}\ko}	
	\mpassive{Choco Mend}{Whenever you are not in combat, you can spend 10 minutes of time to cure an ally from any Status Effect except KO.}
}
%
\vfill
%
\ofmonster{White Chocobo}{1}{\includegraphics[width=0.23\columnwidth]{./art/chocobo/chocwhite.jpg}}
{
	HP: & \hfill 18 & MP: & \hfill 24\\
	STR: & \hfill 1 & DEF: & \hfill 0 \\
	MAG: & \hfill 1 & RES: & \hfill 1 \\
	AGI: & \hfill 2 & STA: & \hfill 3 \\
}
{\accf{Beak}: 1d DMG \hfill \accf{Resilient:}\holy}
{	
	\mspell{Haste (Level 1)}{8}{0r}{Single}{3u}{The target gains Haste for 3 rounds.}{\accf{Level 1}}
	\mspell{White Wind (Level 3)}{14}{0r}{4u (line)}{Self}{All allies in the target area regain an amount of HP equal to half of your current HP and are cured of all negative Status Effects except KO.}{\accf{Level 3}}
	\mspell{Recharge (Level 6)}{8}{1r}{3u}{Self}{All allies within the target area except you regain 3d MP.}{\accf{Level 6}}
	\mspell{Holy (Level 9)}{20}{2r}{Single}{7u}{You deal 6d+20 holy damage to the target.}{\accf{Level 9}\holy}
	\mpassive{Choco Sense}{You can sense the presence of hostile monsters in distance of up to 200u.}		
}
%
\vfill
%
\ofquote{"She’ll tell us when she’s ready, so just hold your Chocobos until then, ya?"}{Wakka}
%
\newpage
%
\ofmonster{Black Chocobo}{1}{\includegraphics[width=0.23\columnwidth]{./art/chocobo/chocblack.jpg}}
{
	HP: & \hfill 19 & MP: & \hfill 23\\
	STR: & \hfill 1 & DEF: & \hfill 1 \\
	MAG: & \hfill 1 & RES: & \hfill 0 \\
	AGI: & \hfill 2 & STA: & \hfill 3 \\
}
{\accf{Beak}: 1d DMG \hfill \accf{Resilient:}\dark}
{
	\mspell{Gravity (Level 1)}{6}{0r}{Single}{3u}{The target suffers 2d damage and can only move half his usual distance on his next turn. 
	}{\accf{Level 1}}
	\mspell{Petrify (Level 3)}{7}{1r}{Single}{5u}{The target makes a DC 8 and suffers Immobile for 3 rounds upon failure.}{\accf{Level 3}\immobile}
	\mspell{Imperil (Level 6)}{10}{1r}{Single}{5u}{The target suffers DeDEF and DeRES for 3 rounds}{\accf{Level 6}\dedef \deres}	
	\mspell{Ultima (Level 9)}{25}{2r}{2u}{7u}{Deal 6d+35 dark damage to all enemies in the target area.}{\accf{Level 9}\dark}
	\mpassive{Choco Fly}{You can fly up to 50u above the ground.}		
}
%
\vfill
%
\ofmonster{Golden Chocobo}{1}{\includegraphics[width=0.23\columnwidth]{./art/chocobo/chocgold.jpg}}
{
	HP: & \hfill 25 & MP: & \hfill 35\\
	STR: & \hfill 1 & DEF: & \hfill 1 \\
	MAG: & \hfill 1 & RES: & \hfill 1 \\
	AGI: & \hfill 3 & STA: & \hfill 3 \\
}
{\accf{Beak}: 2d DMG \hfill \accf{All-Immune}}
{
	\mtech{Shine (Level 1)}{5}{0r}{3u}{Self}{All enemies in the target area perform a DC 7 check and suffer Blind for 2 rounds upon failure.}{\accf{Level 1}\blind}
	\mtech{Good Breath (Level 3)}{8}{0r}{3u (front)}{Self}{Remove all Status Effects except KO from all allies in the target area.}{\accf{Level 3}}	
	\mspell{Diaga (Level 5)}{14}{1r}{Single}{6u}{You deal 6d holy damage to the target.}{\accf{Level 5}\fire}
	\mspell{Curaja (Level 7)}{20}{2r}{3u}{5u}{All allies in the target area regain 6d+15 HP.}{\accf{Level 7}}	
	\mtech{Choco Meteor (Level 8)}{27}{2r}{3u}{10u}{Everyone in the target area suffers 6d+40 damage.}{\accf{Level 8}}	
	\mtech{Final Phoenix (Level 10)}{30}{2r}{3u}{Self}{Remove KO from all allies in the target area and fully restore their HP.}{\accf{Level 10}}	
	\mpassive{Choco Sense}{You can sense the presence of hostile monsters in distance of up to 200u.}		
	\mpassive{Choco Fly}{You can fly up to 50u above the ground.}	
}
%
\clearpage 
\monster{Bomb}{3}{\includegraphics[width=0.17\textwidth]{./art/monsters/bomb.png}}
{
	HP: & \hfill 22 & MP: & \hfill 12\\
	STR: & \hfill 3 & DEF: & \hfill 2 \\
	MAG: & \hfill 2 & RES: & \hfill 1 \\
	AGI: & \hfill 3 & Size: & \hfill M\\
}
{
	\textbf{Tackle}: 1d DMG \hfill \textbf{Drops:} 200 Gil \\
	\textbf{Resilient}:\fire \hfill \textbf{Weak}:\ice 

	\mtech{Self-Destruct}{0}{1r}{2u}{Self}{
		Inflict \hyperlink{status}{KO} on yourself to deal 6d \hyperlink{type}{fire} damage to everyone within the target area.
	}{\fire}		
	%\vspace{0.1cm} \hrule \vspace{0.1cm} 
	%\emph{"Run run, or you'll be well done!" -- Kefka}
} 
\monster{Blue Flan}{3}{\includegraphics[width=0.25\textwidth]{./art/monsters/flan.png}}
{
	HP: & \hfill 12 & MP: & \hfill 30\\
	STR: & \hfill 0 & DEF: & \hfill 6 \\
	MAG: & \hfill 5 & RES: & \hfill 1 \\
	AGI: & \hfill 1 & Size: & \hfill M\\
}
{
	\textbf{Tackle}: 1d DMG \hfill 	\textbf{Drops:} 200 Gil \\
	\textbf{Resilient}:\ice \hspace*{\fill} \textbf{Weak}:\fire 
	
	\mspell{Blizzard}{4}{1r}{Single}{3u}{
		You deal 2d \hyperlink{type}{ice} damage to the target.
	}{\ice}		
	%\vspace{0.1cm} \hrule \vspace{0.1cm} 
	%\emph{"Icing on the cake!" -- Lulu}
}
 
\monster{Abeja Asesina}{3}{\includegraphics[width=0.15\textwidth]{./art/monsters/killerbee.png}}
{
 PV: & \hfill 18 & PM: & \hfill 0 \\
 FUE: & \hfill 2 & DEF: & \hfill 1 \\
 MAG: & \hfill 1 & RES: & \hfill 3 \\
 AGI: & \hfill 3 & Tamaño: & \hfill P\\
}
{
 \textbf{Aguijón}: 2d de daño \hfill \textbf{Botín:} 150 Gil \\
 \textbf{Inmune}:\poison \mpassive{Toque Venenoso}{
 Siempre que hagas un \hyperlink{action}{Ataque} con éxito, el objetivo debe hacer una tirada con DC 8. Si falla, queda \hyperlink{status}{Envenenado} por 3 turnos.
	}
	%\vspace{0.1cm} \hrule \vspace{0.1cm} 
 %\emph{"¡No me gusta tener criaturas de bajo vuelo tratando de matarme!"\\ -- Sazh}
}  

%%%%%%%%%%%%%%L4%%%%%%%%%%%%%%
\monster{Gárgola}{4}{\includegraphics[width=0.23\textwidth]{./art/monsters/gargoyle.png}}
{
 PV: & \hfill 35 & PM: & \hfill 0\\
 FUE: & \hfill 4 & DEF: & \hfill 6 \\
 MAG: & \hfill 0 & RES: & \hfill 1 \\
 AGI: & \hfill 2 & Tamaño: & \hfill M\\
}
{
 \textbf{Garra}: 2d de daño \hfill \textbf{Botín:} 300 Gil \\
 \textbf{Resistencia}:\earth \hfill \textbf{Debilidad}:\water %\vspace{0.1cm} \hrule \vspace{0.1cm} 
 %\emph{"Ese muchacho no es muy optimista." -- Alisaie}
}
\monster{Ahriman}{4}{\includegraphics[width=0.21\textwidth]{./art/monsters/ahriman.png}}
{
	HP: & \hfill 28 & MP: & \hfill 24\\
	STR: & \hfill 0 & DEF: & \hfill 1 \\
	MAG: & \hfill 4 & RES: & \hfill 4 \\
	AGI: & \hfill 4 & Size: & \hfill S\\
}
{
	\textbf{Beam}: 2d DMG, 3u Range \hfill \textbf{Drops:} 350 Gil 
	
	\mtech{Eerie Soundwave}{6}{1r}{Single}{3u}{
		The target makes a DC 8 check and suffers  2d damage and \hyperlink{status}{Silence} for 3 rounds upon failure.
	}{\silence}		
	%\vspace{0.1cm} \hrule \vspace{0.1cm} 
	%\emph{"There are none who can stop me in all of creation! You, too shall fall before me!" -- Ahriman}	
} 
\monster{Antlion}{4}{\includegraphics[width=0.26\textwidth]{./art/monsters/antlion.png}}
{
	HP: & \hfill 40 & MP: & \hfill 16\\
	STR: & \hfill 3 & DEF: & \hfill 3 \\
	MAG: & \hfill 0 & RES: & \hfill 1 \\
	AGI: & \hfill 3 & Size: & \hfill M\\
}
{
	\textbf{Bite}: 2d DMG \hfill \textbf{Drops:} 300 Gil \\
	\textbf{Resilient}:\earth \hfill \textbf{Immune}:\blind 
	
	\mtech{Sandstorm}{8}{1r}{3u}{Self}{
		All enemies in the target area make a DC 9 check and suffer 3d \hyperlink{type}{earth} damage and \hyperlink{status}{Blind} for 3 rounds upon failure.
	}{\earth \blind}		
	%\vspace{0.1cm} \hrule \vspace{0.1cm} 
	%\emph{"It's okay. Antlions are quite tame." -- Edward}
}
 
\monster{Diablillo}{4}{\includegraphics[width=0.13\textwidth]{./art/monsters/imp.png}}
{
 PV: & \hfill 30 & PM: & \hfill 50 \\
 FUE: & \hfill 1 & DEF: & \hfill 2 \\
 MAG: & \hfill 5 & RES: & \hfill 4 \\
 AGI: & \hfill 4 & Tamaño: & \hfill P\\
}
{
 \textbf{Garra}: 2d de daño \hfill \textbf{Botín:} 400 Gil \\
 \textbf{Resistencia}:\dark \mspell{Confusión}{10}{1t}{Único}{5u}{
 El objetivo hace una tirada con DC 8. Si falla, queda bajo tu control en su próximo turno. Puedes ordenarle que se mueva y que \hyperlink{action}{Ataque} a cualquier objetivo que elijas, incluyendo a él mismo. }{} %\vspace{0.1cm} \hrule \vspace{0.1cm} 
 %\emph{"Quizás Dios perdonaría a un horrible pedazo de basura como tú... ¡pero yo no!" -- Cid}
} 
\monster{Mummy}{4}{\includegraphics[width=0.14\textwidth]{./art/monsters/mummy.png}}
{
	HP: & \hfill 38 & MP: & \hfill 0\\
	STR: & \hfill 2 & DEF: & \hfill 3 \\
	MAG: & \hfill 0 & RES: & \hfill 1 \\
	AGI: & \hfill 2 & Size: & \hfill M\\
}
{
	\textbf{Bite}: 2d DMG \hfill \textbf{Drops:} 350 Gil  \\
	\textbf{Immune}:\poison\sleep \hfill \textbf{Weak}:\fire 
	
	\mpassive{Zombietouch}{Whenever you successfully \hyperlink{action}{Attack} a target he makes a DC 8 check and suffers \hyperlink{status}{Zombie} for 1~hour upon failure.}
	\mpassive{Undead}{You permamently suffer the \hyperlink{status}{Zombie} status.}
	%\vspace{0.1cm} \hrule \vspace{0.1cm} 
	%\emph{"I don't like two-legged things." -- Red XIII}
}		 

%%%%%%%%%%%%%%L5%%%%%%%%%%%%%%
\monster{Drago}{5}{\includegraphics[width=0.25\textwidth]{./art/monsters/wyvern.png}}
{
 PV: & \hfill 45 & PM: & \hfill 50 \\
 FUE: & \hfill 3 & DEF: & \hfill 3 \\
 MAG: & \hfill 2 & RES: & \hfill 3 \\
 AGI: & \hfill 3 & Tamaño: & \hfill M\\
}
{
 \textbf{Garra}: 2d de daño \hfill \textbf{Botín:} 400 Gil 
 
 \mspell{Aero}{8}{1t}{Único}{4u}{
 Infliges 3d de daño de \hyperlink{type}{Viento} al objetivo. }{\wind}
 \mpassive{Zambullida}{
 Siempre que realices un \hyperlink{action}{Ataque} sobre un objetivo, él debe hacer una tirada con DC 6. Si falla, queda \hyperlink{status}{Inmóvil} por 1 turno.
	}
	%\vspace{0.1cm} \hrule \vspace{0.1cm} 
 %\emph{"Y así, Laguna corre por su preciada vida." -- Kiros}
} 
\monster{Quimera}{5}{\includegraphics[width=0.26\textwidth]{./art/monsters/chimera.png}}
{
 PV: & \hfill 60 & PM: & \hfill 90\\
 FUE: & \hfill 1 & DEF: & \hfill 2 \\
 MAG: & \hfill 4 & RES: & \hfill 3 \\
 AGI: & \hfill 3 & Tamaño: & \hfill M\\
}
{
 \textbf{Garra}: 2d de daño \hfill \textbf{Botín:} 500 Gil\\
 \textbf{Resistencia}:\fire \ice \lightning 
 
 \mspell{Piro++}{12}{2t}{Único}{5u}{Inflijes 6d de daño de \hyperlink{type}{Fuego} al objetivo. }{\fire} 
 \mspell{Hielo++}{12}{2t}{Único}{5u}{Inflijes 6d de daño de \hyperlink{type}{Hielo} al objetivo. }{\ice} 
 \mspell{Electro++}{12}{2t}{Único}{5u}{Inflijes 6d de daño de \hyperlink{type}{Eléctrico} al objetivo.}{\lightning}
 %\vspace{0.1cm} \hrule \vspace{0.1cm} 
 %\emph{"Genial, ahora estoy luchando contra cuentos de hadas." -- Lightning}
} 
\monster{Gigas}{5}{\includegraphics[width=0.13\textwidth]{./art/monsters/gigas.png}}
{
	HP: & \hfill 70 & MP: & \hfill 50\\
	STR: & \hfill 6 & DEF: & \hfill 4 \\
	MAG: & \hfill 1 & RES: & \hfill 2 \\
	AGI: & \hfill 1 & Size: & \hfill L\\
}
{
	\textbf{Fist}: 2d DMG, 2u Range \hfill \textbf{Drops:} 450 Gil
	
	\mtech{Headbutt}{10}{1r}{Single}{2u}{
		You deal 6d damage to the target and knock him back by 3u.
	}{}		
	%\vspace{0.1cm} \hrule \vspace{0.1cm} 
	%\emph{"Ugh! I hate muscle men! What, are you the backup?" -- Ultros}	
} 
\friendly{Magic Pot}{5}{\includegraphics[width=0.12\textwidth]{./art/monsters/magicpot.png}}
{
	HP: & \hfill 1 & MP: & \hfill 1\\
	STR: & \hfill 0 & DEF: & \hfill 99 \\
	MAG: & \hfill 0 & RES: & \hfill 99 \\
	AGI: & \hfill 1 & Size: & \hfill S\\
}
{
	\textbf{Immune}: \hyperlink{status}{All Status Effects} \hfill 	\textbf{Drops:} 1000 Gil 
	
	\mreaction{Gimme!}{
		When given a beneficial \hyperlink{item}{item} you disappear (\hyperlink{status}{KO}), dropping Gil.
		Otherwise, you make a DC 8 check and upon failure you suffer \hyperlink{status}{KO} to deal 8d damage in 3u around you, dropping no Gil.
	}
	%%%%%%%%%%%%%%%%%%%%%%%%%%%%%%%%%%%%%%%%%%
	%\vspace{0.1cm} \hrule \vspace{0.1cm} 
	%\emph{"Gimme Elixir!" -- Magic Pot}
}
\monster{Cactilio}{5}{\includegraphics[width=0.14\textwidth]{./art/monsters/cactuar.png}}
{
 PV: & \hfill 20 & PM: & \hfill 40\\
 FUE: & \hfill 1 & DEF: & \hfill 1 \\
 MAG: & \hfill 4 & RES: & \hfill 10 \\
 AGI: & \hfill 6 & Tamaño: & \hfill P\\
}
{
 \textbf{Tacleo:} 1d de daño \hfill \textbf{Botín:} 1000 Gil \\
 \textbf{Inmune}:\sleep \blind \mtech{1000 Agujas}{10}{0t}{Único}{1u}{
 Infliges 10d de daño al objetivo. }{} \mpassive{Huída}{Cuando estés huyendo de tus enemigos, puedes moverte 2u más.} 
	%%%%%%%%%%%%%%%%%%%%%%%%%%%%%%%%%%%%%%%%%%
	\vspace{0.1cm} \hrule \vspace{0.1cm} 
 "Agujas. ¡Odio las agujas!" -- Rikku }
 

%%%%%%%%%%%%%%L6%%%%%%%%%%%%%%
\monster{Mindflayer}{6}{\includegraphics[width=0.20\textwidth]{./art/monsters/mindflayer.png}}
{
	HP: & \hfill 65 & MP: & \hfill 130 \\
	STR: & \hfill 1 & DEF: & \hfill 2 \\
	MAG: & \hfill 6 & RES: & \hfill 7 \\
	AGI: & \hfill 2 & Size: & \hfill M\\
}
{
	\textbf{Staff}: 1d DMG \hfill \textbf{Drops:} 700 Gil \\
	\textbf{Weak}:\lightning \hfill	\textbf{Resilient}:\water \\
	\textbf{Immune}:\poison\silence\sleep
	
	\mspell{Waterga}{14}{2r}{Single}{5u}{
		You deal 8d \hyperlink{type}{water} damage to the target.  
	}{\water}
	\mtech{Mind Blast}{20}{1r}{2u}{5u}{
		All enemies in the target area suffer 4d \hyperlink{type}{dark} damage and \hyperlink{status}{Immobile} for 1 round.
	}{\immobile \dark}	
	%\vspace{0.1cm} \hrule \vspace{0.1cm} 
	%\emph{"Why do I get the feeling this is not the safest place to be...?" -- Luneth}	
}
 
\monster{Medusa}{6}{\includegraphics[width=0.18\textwidth]{./art/monsters/medusa.png}}
{
	HP: & \hfill 70 & MP: & \hfill 110\\
	STR: & \hfill 3 & DEF: & \hfill 4 \\
	MAG: & \hfill 5 & RES: & \hfill 4 \\
	AGI: & \hfill 3 & Size: & \hfill M\\
}
{
	\textbf{Hair}: 2d DMG \hfill \textbf{Drops:} 750 Gil \\
	\textbf{Resilient}:\earth\lightning \hfill \textbf{Weak}:\water \\
	\textbf{Immune}:\poison\sleep\immobile 
	
	
	\mtech{Gaze}{15}{1r}{3u (front)}{Self}{Everyone in the target area makes a DC~8 check~and suffers \hyperlink{status}{Immobile} for 3 rounds upon failure.}{\immobile}	
	\mspell{Thundaga}{12}{2r}{Single}{5u}{You deal 6d \hyperlink{type}{lightning} damage to the target.}{\lightning}
	\vspace{0.1cm} \hrule \vspace{0.1cm} 
	"Just lookin' at you is makin' me sober." -- Reno
} 
\monster{Lamia}{6}{\includegraphics[width=0.24\textwidth]{./art/monsters/lamia.png}}
{
 PV: & \hfill 70 & PM: & \hfill 100 \\
 FUE: & \hfill 2 & DEF: & \hfill 3 \\
 MAG: & \hfill 6 & RES: & \hfill 5 \\
 AGI: & \hfill 3 & Tamaño: & \hfill M\\
}
{
 \textbf{Bofetada}: 2d de daño \hfill \textbf{Botín:} 800 Gil \\
 \textbf{Resistencia}:\water \hfill \textbf{Débil}:\lightning \\
 \textbf{Inmune}:\poison\sleep\silence \mspell{Rana}{16}{2t}{Único}{5u}{
 El objetivo debe hacer una tirada con DC 8. Si falla, queda convertido en una rana por 5 turnos o hasta que reciba daño. Mientras esté convertido en rana, el objetivo no puede hablar ni realizar ninguna acción y solo puede moverse 1u por turno. }{} \mreaction{Encantar}{
 Siempre que un enemigo te golpee con éxito con un \hyperlink{action}{Ataque}, debe hacer una tirada con DC~6.  Si falla, puedes ordenarle que realice las acciones y movimientos que elijas y debe seguir esas órdenes en su próximo turno. 
	}
	%\vspace{0.1cm} \hrule \vspace{0.1cm} 
 %\emph{"¿¡Ranas!? ¡Odio las ranas! ¡No me conviertas en una!" -- Refia}
} 
\monster{Iron Giant}{6}{\includegraphics[width=0.16\textwidth]{./art/monsters/irongiant.png}}
{
	HP: & \hfill 80 & MP: & \hfill 48 \\
	STR: & \hfill 5 & DEF: & \hfill 5 \\
	MAG: & \hfill 0 & RES: & \hfill 4 \\
	AGI: & \hfill 3 & Size: & \hfill L\\
}
{
	\textbf{Sword}: 3d DMG, 2u Range \hfill \textbf{Drops:} 500 Gil %\\
	%\textbf{Resilient}:\physical  
	
	\mtech{Sweep}{12}{1r}{3u (front)}{Self}{
		Make an \hyperlink{action}{Attack} against all enemies in target area.
	}{}		
} 

%%%%%%%%%%%%%%L7%%%%%%%%%%%%%%
\monster{\hypertarget{malboro}{Malboro}}{7}{\includegraphics[width=0.21\textwidth]{./art/monsters/malboro.png}}
{
 PV: & \hfill 100 & PM: & \hfill 100 \\
 FUE: & \hfill 2 & DEF: & \hfill 4 \\
 MAG: & \hfill 5 & RES: & \hfill 7 \\
 AGI: & \hfill 2 & Tamaño: & \hfill G\\
}
{
 \textbf{Tentáculo}: 2d de daño \hfill \textbf{Botín:} 1000 Gil \\
 \textbf{Inmune}: \hyperlink{status}{Todos los Estados Alterados} \hfill \textbf{Debilidad}:\fire 
 
 \mtech{Mal Aliento}{20}{1t}{3u (frente)}{Tú}{
 Todos los enemigos en el área de efecto hacen una tirada con DC 8. Si fallan, quedan \hyperlink{status}{Dormidos},   \hyperlink{status}{Envenenados}, en \hyperlink{status}{Silencio} y \hyperlink{status}{Ciegos} por 3 turnos. }{\sleep \poison \silence \blind} \mtech{Jugo Gástrico}{10}{1t}{2u}{8u}{
 Todos los enemigos dentro del área de efecto reciben 5d de daño y deben hacer una tirada con DC 8. Todos los que fallen reciben  \hyperlink{status}{disFUE} y \hyperlink{status}{disMAG} por 5 turnos. }{\destr \demag} %\vspace{0.1cm} \hrule \vspace{0.1cm} 
 %\emph{"Eso parece una boca. ¿¡Es esa su cara!?" -- Prompto}
}
 
\monster{Zu}{7}{\includegraphics[width=0.23\textwidth]{./art/monsters/zu.png}}
{
	HP: & \hfill 160 & MP: & \hfill 100 \\
	STR: & \hfill 7 & DEF: & \hfill 5 \\
	MAG: & \hfill 3 & RES: & \hfill 6 \\
	AGI: & \hfill 2 & Size: & \hfill L\\
}
{
	\textbf{Claw}: 3d DMG \hfill \textbf{Drops:} 1000 Gil \\
	\textbf{Immune}:\poison \sleep \silence
		
	\mtech{Tornado}{20}{1r}{9u (line)}{Self}{
			You create a tornado with a 2u diameter that travels 3u in a line per round for the next 3 rounds.
			Anyone except you that gets into contact with it suffers 4d \hypertarget{type}{wind} damage and \hyperlink{status}{Immobile} for 1 round.	
	}{\wind \immobile}	
	%\mpassive{Auto-Regen}{
	%	You regain 25 HP at the start of each turn.
	%}	
	%\vspace{0.1cm} \hrule \vspace{0.1cm} 
	%\emph{"How can a bird grow so big?" -- Tidus}	
}
 
\monster{Cerberus}{7}{\includegraphics[width=0.21\textwidth]{./art/monsters/cerberus.png}}
{
	HP: & \hfill 120 & MP: & \hfill 110 \\
	STR: & \hfill 5 & DEF: & \hfill 3 \\
	MAG: & \hfill 6 & RES: & \hfill 4 \\
	AGI: & \hfill 3 & Size: & \hfill L\\
}
{
	\textbf{Bite}: 3d DMG  \\
	\textbf{Resilient}:\fire\ice\lightning \hfill \textbf{Drops:} 1000 Gil 
	%\hfill \textbf{Weak}:\lightning 
	%\textbf{Immune}:\poison\sleep
	
	\mspell{Firaga}{12}{2r}{Single}{5u}{You deal 6d \hyperlink{type}{fire} damage to the target. }{\fire}
	\mpassive{Triple Triad}{You can perform each action on up to 3 different targets within its range simultaneously.}
	%\vspace{0.1cm} \hrule \vspace{0.1cm} 
	%\emph{"So, you've managed to reach these depths. I commend you. But your journey ends here, I'm afraid." -- Cerberus}
} 
\monster{\hypertarget{abyssworm}{Sand Worm}}{7}{\includegraphics[width=0.21\textwidth]{./art/monsters/abyssworm.png}}
{
	HP: & \hfill 150 & MP: & \hfill 120 \\
	STR: & \hfill 7 & DEF: & \hfill 4 \\
	MAG: & \hfill 5 & RES: & \hfill 6 \\
	AGI: & \hfill 1 & Size: & \hfill L\\
}
{
	\textbf{Acid}: 3d DMG, 3u Range \hfill \textbf{Drops:} 1000 Gil \\
	\textbf{Immune}:\poison \sleep 
	
	\mspell{Quake}{22}{2r}{4u}{8u}{
		You deal 8d \hyperlink{type}{earth} damage to everyone in the target area. 
	}{\earth}	
	\mtech{Inhale}{20}{1r}{Single}{3u}{
		You inhale the target, removing him from the battle. 
		At the beginning of every turn he may try to free himself by passing a DC 9 check.
	}{}		
	%\vspace{0.1cm} \hrule \vspace{0.1cm} 
	%\emph{"Ah, where’s the early bird when you need one?" -- Wakka}	
}
 
\monster{Zombie Dragon}{7}{\includegraphics[width=0.28\textwidth]{./art/monsters/zombiedragon.png}}
{
	HP: & \hfill 125 & MP: & \hfill 80\\
	STR: & \hfill 6 & DEF: & \hfill 5 \\
	MAG: & \hfill 0 & RES: & \hfill 3 \\
	AGI: & \hfill 2 & Size: & \hfill L\\
}
{
	\textbf{Bite}: 3d DMG, 2u Range \hfill	\textbf{Drops:} 900 Gil  \\
	\textbf{Immune}:\ko\poison\sleep\silence \hfill  \textbf{Weak}:\holy
  
	
	\mtech{Poison Breath}{10}{1r}{3u (front)}{Self}{Everyone in the target area suffers 4d damage, makes a DC~8 check~and suffers \hyperlink{status}{Poison} for 3 rounds upon failure.}{\poison}	
	\mreaction{Regenerate}{
		When reduced below 50 HP, you become unable to move or act.
		You regenerate 50 HP on each turn for 3 rounds after which you can act and move again.  
		This effect can only be used once per battle.
	}	
	\mpassive{Undead}{You permamently suffer the \hyperlink{status}{Zombie} status.}
	%\vspace{0.1cm} \hrule \vspace{0.1cm} 
	%\emph{"Don't eat me! I won't taste good!" -- Eiko}
}
 

%%%%%%%%%%%%%%L8%%%%%%%%%%%%%%
\monster{Behemoth}{8}{\includegraphics[width=0.32\textwidth]{./art/monsters/behemoth.png}}
{
 PV: & \hfill 200 & PM: & \hfill 250 \\
 FUE: & \hfill 8 & DEF: & \hfill 5 \\
 MAG: & \hfill 5 & RES: & \hfill 4 \\
 AGI: & \hfill 3 & Tamaño: & \hfill G\\
}
{
 \textbf{Garra}: 3d de daño, 2u Alcance \hfill \textbf{Botín:} 1500 Gil \\
 \textbf{Inmune}:\poison \silence 
 
 \mspell{Llamarada}{30}{3t}{Único}{5u}{Infliges 9d+15 de daño de \hyperlink{type}{Fuego} al objetivo. }{\fire}
 \mtech{Embestida}{20}{0t}{Único}{2u}{Infliges 10d de daño al objetivo y lo haces volar 3u por el aire durante 1 turno. }{} 
 \mreaction{Contraataque}{Siempre que seas el objetivo de la acción de un enemigo que se encuentre a 2u de ti, inmediatamente haz un \hyperlink{action}{Ataque} sobre él.}
%\vspace{0.1cm} \hrule \vspace{0.1cm} 
%\emph{"¿Suficientemente grande para ti?" -- Gladio} 
}
 
\monster{Ochu}{8}{\includegraphics[width=0.27\textwidth]{./art/monsters/ochu.png}}
{
 PV: & \hfill 180 & PM: & \hfill 100 \\
 FUE: & \hfill 7 & DEF: & \hfill 5 \\
 MAG: & \hfill 6 & RES: & \hfill 4 \\
 AGI: & \hfill 1 & Tamaño: & \hfill G\\
}
{
 \textbf{Cepa}: 3d de daño, 3u Alcance \hfill \textbf{Botín:} 1500 Gil \\
 \hfill \textbf{Inmune}:\poison \sleep \blind \mtech{Polen}{15}{1t}{5u}{Tú}{
 Todos los enemigos que se encuentren en el área de efecto hacen una tirada con DC 8. Si fallan, quedan \hyperlink{status}{Dormidos} y \hyperlink{status}{Envenenados} por 3 turnos. }{\sleep \poison} %\vspace{0.1cm} \hrule \vspace{0.1cm} 
 %\emph{"El Ochu no es un demonio común y corriente. Podríamos atacarlo con cien Cruzados y aún así perderíamos." -- Luzzu}
} 
\monster{Demon Wall}{8}{\includegraphics[width=0.17\textwidth]{./art/monsters/demonwall.png}}
{
	HP: & \hfill 250 & MP: & \hfill 200 \\
	STR: & \hfill 9 & DEF: & \hfill 6 \\
	MAG: & \hfill 5 & RES: & \hfill 6 \\
	AGI: & \hfill 2 & Size: & \hfill L\\
}
{
	\textbf{Swords}: 3d DMG, 2u Range \hfill  \textbf{Drops:} 2000 Gil \\
	\textbf{Immune}: \hyperlink{status}{All Status Effects} 
	
	\mtech{Sleepga}{24}{2r}{2u}{5u}{
		All enemies within the target area make a DC 8 check and suffer \hyperlink{status}{Sleep} upon failure. 
	}{\sleep}	
	\mtech{Wall Rush}{20}{1r}{10u (line)}{Self}{
		You charge forward in a line for up to 10u, dealing 8d damage to everyone in the path and knocking them back by 3u.
		If you crush an enemy between yourself and a wall, they instantly suffer \hyperlink{status}{KO}.
	}{\ko}		
	\mpassive{Heavy Turn}{
		Every time you change direction, everyone within 2u suffers 6d damage, but you cannot take an action on the same turn.
	}
	\vspace{0.1cm} \hrule \vspace{0.1cm} 
	"As if the attacking doors weren't enough..." -- Rydia
}

 
\monster{Dragón Rojo}{8}{\includegraphics[width=0.2\textwidth]{./art/monsters/reddragon.png}}
{
 PV: & \hfill 225 & PM: & \hfill 250 \\
 FUE: & \hfill 6 & DEF: & \hfill 4 \\
 MAG: & \hfill 5 & RES: & \hfill 4 \\
 AGI: & \hfill 3 & Tamaño: & \hfill G\\
}
{
 \textbf{Mordida}: 3d de daño \hfill \textbf{Botín:} 1500 Gil \\
 \textbf{Resistencia}:\fire \hfill \textbf{Débil}:\ice \\
 \textbf{Inmune}:\sleep\blind\immobile \mspell{Fulgor}{30}{3t}{Único}{5u}{Infliges 9d+15 de daño de \hyperlink{type}{Fuego} al objetivo. }{\fire}
 \mtech{Llamarada}{22}{1t}{3u (frente)}{Tú}{
 Infliges 8d de daño de \hyperlink{type}{Fuego} a todos los enemigos que se encuentren dentro del área de efecto. }{\fire}
 \mpassive{Coletazo}{ Cuando realices un \hyperlink{action}{Ataque}, puedes elegir atacar a todos los enemigos que se encuentren a 1u de ti.}
	%\vspace{0.1cm} \hrule \vspace{0.1cm} 
 %\emph{"¡No sabía que un dragón tan poderoso existía en este universo!" -- Kain} 
} 

%%%%%%%%%%%%%%L9%%%%%%%%%%%%%%
\monster{Tonberry}{9}{\includegraphics[width=0.26\textwidth]{./art/monsters/tonberry.png}}
{
	HP: & \hfill 280 & MP: & \hfill 100\\
	STR: & \hfill 12 & DEF: & \hfill 6 \\
	MAG: & \hfill 3 & RES: & \hfill 5 \\
	AGI: & \hfill 2 & Size: & \hfill S\\
}
{
	\textbf{Knife}: 4d DMG \hfill \textbf{Drops:} 4000 Gil \\
	\textbf{Immune}:\poison \blind \sleep  
		
	\mpassive{Grudge}{
		Every time you \hyperlink{status}{Attack} an enemy, he has to make a DC 7 check and suffers \hyperlink{status}{KO} upon failure. 
	}	
	\mreaction{Karma}{Whenever an enemy that is more than 3u away reduces your HP, deal 8d \hyperlink{type}{dark} damage back. }
	\vspace{0.1cm} \hrule \vspace{0.1cm} 
	"A kitchen knife. Wonder if it's a culinary battle he wants." -- Ignis
} 
\monster{Kraken}{9}{\includegraphics[width=0.23\textwidth]{./art/monsters/kraken.png}}
{
	HP: & \hfill 320 & MP: & \hfill 400 \\
	STR: & \hfill 7 & DEF: & \hfill 5 \\
	MAG: & \hfill 11 & RES: & \hfill 7 \\
	AGI: & \hfill 2 & Size: & \hfill L\\
}
{
	\textbf{Tentacle}: 4d DMG, 2u range \hfill \textbf{Drops:} 2000 Gil \\
	\textbf{Resilient}:\water\ice \hfill \textbf{Immune}:\poison\sleep\blind 
	
	\mspell{Waterga}{14}{2r}{Single}{5u}{You deal 8d \hyperlink{type}{water} damage to the target. }{\water}	
	\mtech{Ink}{22}{1r}{2u}{5u}{
		All enemies within the target area make a DC 8 check and suffer \hyperlink{status}{Blind} and 4d damage upon failure. 
	}{\blind}
	\mpassive{Multiattack}{Whenever you choose to \hyperlink{action}{Attack}, you can target all enemies within range at once.}
	%\vspace{0.1cm} \hrule \vspace{0.1cm} 
	%\emph{"I am the Kraken... Your presence is forbidden!"}
} 
\monster{Adamantoise}{9}{\includegraphics[width=0.31\textwidth]{./art/monsters/adamantoise.png}}
{
	HP: & \hfill 350 & MP: & \hfill 400 \\
	STR: & \hfill 10 & DEF: & \hfill 7 \\
	MAG: & \hfill 8 & RES: & \hfill 6 \\
	AGI: & \hfill 1 & Size: & \hfill L\\
}
{
	\textbf{Trample}: 4d DMG, 2u Target \hfill \textbf{Drops:} 3000 Gil \\
	\textbf{Immune}:\poison \silence \hfill \textbf{Resilient}:\earth
	
	\mspell{Ultima}{45}{3r}{3u}{5u}{
		You deal 10d+20 \hyperlink{type}{dark} damage to all enemies in the target area.  
	}{\dark}	
	\mtech{Roar}{20}{1r}{2u}{5u}{
		All enemies within the target area make a DC 9 check and suffer \hyperlink{status}{Immobile} for 3 rounds upon failure. 
	}{\immobile}	
	\mreaction{Vigor}{
		Whenever you suffer damage while concentrating, the cast time of the ability you are preparing is reduced by 1 round.  
	}	
	%\vspace{0.1cm} \hrule \vspace{0.1cm} 
	%\emph{"Not our lucky day." -- Fang}	
}
 
\monster{Lich}{9}{\includegraphics[width=0.20\textwidth]{./art/monsters/lich.png}}
{
	HP: & \hfill 300 & MP: & \hfill 500\\
	STR: & \hfill 4 & DEF: & \hfill 5 \\
	MAG: & \hfill 12 & RES: & \hfill 8 \\
	AGI: & \hfill 2 & Size: & \hfill L\\
}
{
	\textbf{Beam}: 4d DMG, 5u Range \hfill \textbf{Drops:} 5000 Gil  \\
	\textbf{Immune}: \hyperlink{status}{All Status Effects}\\
	\textbf{Weak}:\holy \hfill \textbf{Resilient}:\dark 
	
	\mspell{Zombify}{22}{1r}{2u}{5u}{
		Everyone in the target makes a DC 8 check and suffers \hyperlink{status}{Zombie} for 1 hour upon failure.
	}{\zombie}	
	\mspell{Poisonga}{24}{1r}{2u}{5u}{
		Everyone in the target area makes a DC 8 check and suffers \hyperlink{status}{Poison} for 3 rounds upon failure.
	}{\poison}	
	\mspell{Doom}{36}{1r}{Single}{5u}{
		The target makes a DC 8 check and suffers \hyperlink{status}{KO} after 3 rounds upon failure.
	}{\ko}	
	\mpassive{Greater Undead}{
		You permamently suffer \hyperlink{status}{Zombie}, but are immune to effects that cause or cure \hyperlink{status}{KO}.
	}
	%\vspace{0.1cm} \hrule \vspace{0.1cm} 
	%\emph{"Was that Death himself? I felt a sudden dread course through my veins..." -- Ceodore}
}

 

%%%%%%%%%%%%%%L10%%%%%%%%%%%%%%
\monster{Caos}{10}{\includegraphics[width=0.23\textwidth]{./art/monsters/chaos.png}}
{
 PV: & \hfill 400 & PM: & \hfill 700 \\
 FUE: & \hfill 8 & DEF: & \hfill 8 \\
 MAG: & \hfill 10 & RES: & \hfill 7 \\
 AGI: & \hfill 4 & Tamaño: & \hfill G\\
}
{
 \textbf{Haz}: 5d de daño, 3u Alcance \hfill \textbf{Botín:} 10000 Gil \\
 \textbf{Inmune}: \hyperlink{status}{Todos los Estados Alterados} \\
 \textbf{Resistencia}:\dark\fire \mspell{Artema}{45}{3t}{3u}{5u}{
 Infliges 10d+10 de daño \hyperlink{type}{Oscuro} solo a los enemigos que se encuentren en el área de efecto. }{\dark} \mspell{Cura+++}{30}{2t}{3u}{5u}{
 Todos los aliados que se encuentren en el área de efecto recuperan 8d+10 de sus PV. }{} \mspell{Piro+++}{28}{2t}{3u}{8u}{
 Infliges 8d+10 de daño de \hyperlink{type}{Fuego} a todos los que se encuentren en el área de efecto. }{\fire}
 \mpassive{Toque Caótico}{
 Siempre que hagas un \hyperlink{action}{Ataque} con éxito, el objetivo debe hacer una tirada con DC 8. Si falla, sufre \hyperlink{status}{Veneno}, \hyperlink{status}{Ciego} y \hyperlink{status}{Silencio} por 3 turnos.
	}
 \mreaction{Acelerar}{
 Siempre que recibas daño, puedes hacer una tirada con DC 6. Si tienes éxito, tienes un turno adicional inmediatamente después del atacante. Este efecto no cambia el orden habitual de los turnos y solo puede utilizarse una vez por turno.
	}
	\vspace{0.1cm} \hrule \vspace{0.1cm} 
 "Pero renaceré una vez más. Así que aunque muera una y otra vez, yo regresaré. ¡Renacido en este ciclo infinito que he creado!" -- Caos }
 
\monster{Shinryu}{10}{\includegraphics[width=0.16\textwidth]{./art/monsters/shinryu.png}}
{
	HP: & \hfill 500 & MP: & \hfill 800 \\
	STR: & \hfill 12 & DEF: & \hfill 8 \\
	MAG: & \hfill 14 & RES: & \hfill 9 \\
	AGI: & \hfill 4 & Size: & \hfill L\\
}
{
	\textbf{Tail}: 5d DMG, 4u Range \hfill \textbf{Drops:} Dragon Seal\\
	\textbf{Immune}: \hyperlink{status}{All Status Effects} 
	
	\mspell{Tidal Wave}{35}{2r}{12u (front)}{Self}{
		All enemies in target area take 9d+10 \hyperlink{type}{water} damage and suffer \hyperlink{status}{Immobile} for 2 rounds.
	}{\water \immobile}	
	\mspell{Atomic Rays}{45}{2r}{8u}{Self}{
		All enemies in the target area take 8d+9 \hyperlink{type}{fire} damage and suffer \hyperlink{status}{Poison} for 3 rounds. 
	}{\fire \poison}		
	\mpassive{Adapt Element}{
		At the start of every turn, choose one element (e.g. \hyperlink{type}{fire}).
		You gain \hyperlink{status}{Resilience} against the element until the start of your next turn.
	}
	\mreaction{Final Attack}{
		When you fall to 0 HP, you can instantly cast a spell without MP cost before suffering \hyperlink{status}{KO}.
	}
	\vspace{0.1cm} \hrule \vspace{0.1cm} 
	"That divine grace... He is something much more than a mere monster." -- Rosa
}
 
\monster{Omega}{???}{\includegraphics[width=0.27\textwidth]{./art/monsters/omega.png}}
{
	HP: & \hfill 999 & MP: & \hfill 999 \\
	STR: & \hfill 14 & DEF: & \hfill 10 \\
	MAG: & \hfill 13 & RES: & \hfill 10 \\
	AGI: & \hfill 5 & Size: & \hfill L\\
}
{
	\textbf{Laser}: 5d DMG, 5u Range \hfill \textbf{Drops:} Omega Badge  \\
	\textbf{Resilient}:\fire\dark\lightning \hfill \textbf{Weak:}\ice\water \\
	\textbf{Immune}: \hyperlink{status}{All Status Effects} 
	
	\mspell{Meltdown}{50}{1r}{10u}{Self}{
		A system vulnerability forces you to leak restricted memory content and lava.
		Deal 8d+25 \hyperlink{type}{fire} damage on the target, including yourself. 
	}{\fire}	
	\mspell{Flamethrower}{12}{0r}{5u (front)}{Self}{
	Deal 6d+10 \hyperlink{type}{fire} damage to all enemies in the target area. 
	}{\fire}
	\mspell{Wave Cannon}{45}{1r}{3u}{12u}{
		You inflict 10d \hyperlink{type}{dark} damage and \hyperlink{status}{DeDEF}, \hyperlink{status}{DeRES} and \hyperlink{status}{DeSTR} for 5 rounds on the target area. 
	}{\dark \dedef \deres}		
	\mpassive{Auto-Repair}{
		You regain 4d HP at the start of every turn. 
	}
	\mreaction{Critical Surge}{
		When your HP is below 10\% of its maximum, you gain \hyperlink{status}{EnSTR}, \hyperlink{status}{EnMAG}, \hyperlink{status}{EnDEF}, \hyperlink{status}{EnRES}.
	}
	\vspace{0.1cm} \hrule \vspace{0.1cm} 
	%%%%%%%%%%%%%%%%%%%%%%%%%%%%%%%%%%%%%%%%%%
	"Man forges a weapon to fell the gods: Omega. 
	The weapon knows nothing of compassion - only destruction!
	Its might knows no equal. 
	The wise dare not cross its path, lest they meet their end."
	\hspace*{0.1cm} -- Gentiana
}
 
