\subsection*{\hypertarget{optrules}{Optional Rules}}
\addcontentsline{toc}{subsection}{Optional Rules}%
%
"Listen up! Teamwork means staying out of my way. It's a Squad B rule."\\
\indent -- Seifer 
%
\begin{center} \includegraphics[width=\columnwidth]{./art/images/ff7.png} \end{center}
%
\vfill
%
You may decide to change the existing rules or add new rules depending on your or the players' preferences.
However, be aware that the game's content is designed around the given rules, there is no guarantee that everything will work well once you modify them.
Therefore, such changes are only recommended to experienced GMs.
Nevertheless, this subsection gives you some examples of interesting rule changes and additions to consider. 

\vfill

\begin{description}[leftmargin=*]

\item[\color{accent} Survival:]
The rules do not focus on realism and survival, which should generally come second to existing fantasy elements.
However, with some small additions you can make your world significantly more unforgiving: 
\begin{itemize}[leftmargin=*]  
	\item The inventory capacity of characters is limited to a total of 10 items or equipment pieces.
	\item Characters who have not eaten properly in one day suffer \hyperlink{status}{DeATR} for all attributes except AGI.
	\item At night or inside unlit areas, characters permanently suffer \hyperlink{status}{Blind}. 
	\item Character who have less than half of their maximum HP have \hyperlink{check}{Disadvantage} on all checks.
\end{itemize}

\vfill

\item[\color{accent} Challenges:]
For some adventures it might be a fun challenge to play with rules that increase the 
difficulty far beyond the usual.
This can be achieved either by limiting the progression of characters or by introducing additional difficulties. 
The rules below give examples for both of these categories:
\begin{itemize}[leftmargin=*]   
	\item A character or monster that receives damage while \hyperlink{status}{KO} is permanently dead.
	\item Characters do not increase any of their attributes at Level up after Level 1.
	\item All negative Status Effects last until they are explicitly removed (e.g. by an \hyperlink{item}{Item}).
\end{itemize}

\pagebreak

\item[\color{accent} Experience Points:]
You can use an experience point system to track character experience instead of the default milestone based scheme. 
In this system, each party member gains one experience point per Level of an enemy defeated in combat (e.g. the party is awarded 10 points when a Level 10 enemy is defeated).
In addition, you can award the party experience points for other achievements such as completing certain tasks.
The table below shows how many total experience points a character needs to reach a certain Level, which you can modify as you see fit. \vspace{0.1cm} \\
\begin{tabular}{@{}p{0.4\columnwidth}@{\hspace{\fill}} r}
	\hspace{0.2cm} Level 1:  &  0 Total Experience Points   \\
	\hspace{0.2cm} Level 2:  &  20 Total Experience Points \\
	\hspace{0.2cm} Level 3:  &  50 Total Experience Points \\
	\hspace{0.2cm} Level 4:  &  100 Total Experience Points  \\
	\hspace{0.2cm} Level 5:  &  175 Total Experience Points \\
	\hspace{0.2cm} Level 6:  &  300 Total Experience Points  \\
	\hspace{0.2cm} Level 7:  &  450 Total Experience Points \\
	\hspace{0.2cm} Level 8:  &  600 Total Experience Points  \\
	\hspace{0.2cm} Level 9:  &  800 Total Experience Points \\
	\hspace{0.2cm} Level 10: & 1000 Total Experience Points  \\
\end{tabular}

\vfill

\item[\color{accent} Quick Battles:]
The following rules allow you to speed up less important battles at the cost of reduced balance:
\begin{itemize}[leftmargin=*]   
	\item All movement is omitted and all combat participants are always within range of each other. 
	Effects that would usually target an area, can either target all allies or all enemies.
	\item Every die on a damage related roll is automatically treated as a 4.
	\item Every battle participants can use their action to try to flee the battle by passing a DC~8 check.
\end{itemize}

\vfill

\item[\color{accent} Back Attack:]
Whenever either the players or their enemies surprise the other party before combat (e.g. with an ambush), they gain an additional surprise round.
If this is the case, everyone in the surprising party takes one turn after rolling for initiative, before the battle starts as usual.

\vfill

\item[\color{accent} Unlimited Progression:]
Characters usually cannot to increase in Levels past a total of 10, but you can remove this limitation to allow for further progression.
In doing so, characters will be able to master multiple jobs and archetypes, while becoming experts in different aspects of combat.	
Accordingly, only the most powerful antagonists will be able to provide a challenge to such a high Level party.

\vfill

\item[\color{accent} Materia:]
Materia are special items that can be used to enchant weapons and armor with additional effects.
However, they are not bound to specific equipment and can be moved around freely at any time to experiment with different setups. 
Each weapon or armor gains one materia slot per equipment Level allowing for multiple special effects on a single equipment piece.

\end{description}

\clearpage

%%%%%%%%%%%%%%%%%%%%%%%%%%%%%%%%%%%%%%%%%
