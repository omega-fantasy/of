\subsection*{\hypertarget{optrules}{Reglas opcionales}}
\addcontentsline{toc}{subsection}{Reglas opcionales}%
%
"¡Escuchen! Trabajar en equipo significa no meterse en mi camino. Es una regla de Equipo B".\\
\indent -- Seifer 
%
\begin{center} \includegraphics[width=\columnwidth]{./art/images/ff7.png} \end{center}
%
\vfill
%
Puedes decidir cambiar las reglas existentes o añadir nuevas reglas según tus preferencias o las de los jugadores. Sin embargo, ten en cuenta que el contenido del juego está diseñado en torno a las reglas dadas y no hay garantía de que todo funcione bien una vez que las modifiques. Por lo tanto, estos cambios solo se recomiendan para los DJ experimentados. Sin embargo, esta subsección te ofrece algunos ejemplos de cambios de reglas interesantes a considerar. 

\vfill

\begin{description}[leftmargin=*]

\item[\color{accent} Supervivencia:] Las reglas no se centran en el realismo y la supervivencia. Éstas quedan en segundo plano frente a los elementos fantásticos existentes. No obstante, con algunos pequeños agregados, puedes hacer que tu mundo sea mucho más implacable: 
\begin{itemize}[leftmargin=*]  
	\item La capacidad de inventario de los personajes se limita a un total de 10 artículos o unidades de equipo.
	\item Los personajes que no han comido debidamente hace un día sufren \hyperlink{status}{disATR} para todos los atributos excepto AGI.
	\item Por la noche o dentro de las áreas poco iluminadas, los personajes sufren permanentemente de \hyperlink{status}{Ceguera}. 
	\item Un personaje que tenga menos de la mitad de sus PV máximos tiene \hyperlink{check}{Desventaja} en todas sus tiradas.
\end{itemize}

\vfill

\item[\color{accent} Desafíos:] Para algunas aventuras, puede ser divertido jugar con reglas que aumenten la dificultad más allá de lo habitual. Para ello, puedes limitar la progresión de los personajes o incluir dificultades adicionales. Las siguientes reglas dan ejemplos de estas dos categorías:
\begin{itemize}[leftmargin=*]   
	\item Un personaje o monstruo que reciba daño mientras está \hyperlink{status}{KO} está permanentemente muerto.
	\item Los personajes no aumentan ninguno de sus atributos después del Nivel 1.
	\item Todos los estados alterados negativos duran hasta que sean explícitamente eliminados (p. ej., por un \hyperlink{item}{Objeto}).
\end{itemize}

\pagebreak

\item[\color{accent} Puntos de experiencia:] Puedes utilizar un sistema de puntos de experiencia para realizar un seguimiento de la experiencia de los personajes en lugar del esquema basado en hitos. En este sistema, cada miembro del grupo gana un punto de experiencia por Nivel del enemigo derrotado en combate (por ejemplo, el grupo recibe 10 puntos cuando se derrota a un enemigo de Nivel 10). Además, puedes otorgarle puntos de experiencia al grupo por otros logros, como completar ciertas tareas. La siguiente tabla muestra cuántos puntos de experiencia total necesita un personaje para alcanzar un nivel determinado. Puedes modificarla como consideres necesario. \vspace{0.1cm} \\
\begin{tabular}{@{}p{0.4\columnwidth}@{\hspace{\fill}} r}
	\hspace{0.2cm} Nivel 1: & 0 puntos de experiencia en total \\
	\hspace{0.2cm} Nivel 2: & 20 puntos de experiencia en total \\
	\hspace{0.2cm} Nivel 3: & 50 puntos de experiencia en total \\
	\hspace{0.2cm} Nivel 4: & 100 puntos de experiencia en total \\
	\hspace{0.2cm} Nivel 5: & 175 puntos de experiencia en total \\
	\hspace{0.2cm} Nivel 6: & 300 puntos de experiencia en total \\
	\hspace{0.2cm} Nivel 7: & 450 puntos de experiencia en total \\
	\hspace{0.2cm} Nivel 8: & 600 puntos de experiencia en total \\
	\hspace{0.2cm} Nivel 9: & 800 puntos de experiencia en total \\
	\hspace{0.2cm} Nivel 10: & 1000 puntos de experiencia en total \\
\end{tabular}

\vfill

\item[\color{accent} Batallas rápidas:] Las siguientes reglas permiten acelerar las batallas menos importantes a costa de un balance reducido:
\begin{itemize}[leftmargin=*]   
	\item Se omite todo movimiento y todos los participantes del combate siempre están a su alcance entre sí. Los efectos que suelen afectar a un área se dirigen a todos los aliados o a todos los enemigos.
	\item Cada tirada de daño se considera que obtuvo un 4.
	\item Todos los participantes de la batalla pueden utilizar su acción para intentar escapar de la batalla pasando una tirada con DC 8.
\end{itemize}

\vfill

\item[\color{accent} Ataque furtivo:] Cada vez que los jugadores o sus enemigos sorprendan a sus contrarios antes de combatir (p. ej., con una emboscada), ganan una ronda sorpresa adicional. Si sucede, todos los miembros del grupo que ataca por sorpresa tienen un turno extra luego de las tiradas de iniciativa antes de que comience la batalla como de costumbre.

\vfill

\item[\color{accent} Niveles ilimitados:] Por lo general, los personajes no pueden subir de Nivel por encima del Nivel 10, pero puedes cambiar esta limitación para permitir una mayor progresión. Al hacerlo, los personajes podrán dominar múltiples oficios y arquetipos, al tiempo que se convertirán en expertos en diferentes aspectos de combate. En consecuencia, solo los antagonistas más potentes podrán proporcionar un desafío a un grupo de nivel alto.

\vfill

\item[\color{accent} Materia:] Las Materias son objetos especiales que se pueden utilizar para encantar armas y armaduras para brindarles efectos adicionales. Sin embargo, no quedan vinculados a un equipo específico y pueden intercambiarse libremente en cualquier momento para experimentar con diferentes combinaciones. Cada arma o armadura obtiene una ranura de Materia por Nivel de equipo, permitiendo múltiples efectos especiales en una sola parte del equipo.

\end{description}

\clearpage

%%%%%%%%%%%%%%%%%%%%%%%%%%%%%%%%%%%%%%%%%
