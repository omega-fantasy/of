\section*{\hypertarget{job}{Oficios}}
\addcontentsline{toc}{subsection}{Oficios}
%
"Fuera de mi silla, bufón. El rey se sienta ahí". \\
\indent -- Noctis 
%
\begin{center}
\includegraphics[width=\columnwidth]{./art/images/ff10-2.png} 
\end{center}
%
El oficio de tu personaje determina sus pericias en combate, incluidas las habilidades, atributos y equipo que pueden utilizar. La maestría en batalla de un personaje aumenta cuando se especializa en un oficio al ganar experiencia y niveles. Todos los oficios disponibles se detallan en la \textbf{Descripción de oficio}, detallada justo después de esta página. Te recomendamos que imprimas o copies la descripción del oficio elegido para usarla como segunda página de tu hoja de personaje. Los atributos de tu personaje empiezan en 0 y aumentan al mejorar en un oficio. La tabla de \textbf{Atributos básicos }muestra los aumentos de los atributos de tu personaje y el equipo que puede utilizar en el Nivel 1. La tabla de \textbf{Habilidades} muestra qué hechizos y técnicas aprenderá tu personaje en los diferentes niveles.

\subsubsection*{Arquetipo}
Cuando tu personaje llegue al \textbf{Nivel 4} en su oficio, debe decidir entre uno de sus dos arquetipos. Los arquetipos representan estilos de juego dentro de un oficio y se pueden considerar como diferentes enfoques para cumplir el mismo rol. Enfatizan el progreso en diferentes aspectos del combate, apoyando las habilidades de un personaje a través de beneficios pasivos. En el Nivel 4, el tipo de arquetipo elegido otorga a tu personaje habilidades \hyperlink{sabilities}{Pasivas y Reactivas}, así como también determina la progresión de atributos en los siguientes niveles.
%
\pagebreak
%
\subsubsection*{Superación de Límites}
A partir del \textbf{Nivel 5}, puedes elegir cualquiera de las habilidades de tu personaje para que sea su habilidad de Superación de Límites, que es una versión mejorada de la habilidad original. Solo puedes cambiar tu habilidad de Superación de Límites cuando subes de nivel. También puedes seguir utilizando la habilidad elegida en su versión original. Cuando utilizas la Superación de Límites, cuesta el doble de PM que de costumbre y obtiene \textbf{uno de los siguientes} efectos adicionales que elijas:
\begin{itemize}[leftmargin=*]  
	\item La cantidad de daño provocada o PV restaurados por la habilidad original se duplicará.
	\item Si la habilidad original afecta a un objetivo único, puedes dirigirlo a dos objetivos dentro de su alcance. Si la habilidad afecta a un área, la distancia máxima de la habilidad se duplicará.
	\item Si la habilidad original tiene un efecto que dura por un tiempo determinado, el mismo se duplicará. 
	\item Si la habilidad original requiere que tú o el objetivo pasen una tirada, puedes aumentar o disminuir la DC en 2 dependiendo de lo que te beneficie. 
\end{itemize}


\subsubsection*{Cambio de Oficio}
Puedes cambiar el oficio de tu personaje solo una vez durante la aventura después de subir de nivel. En lugar de aumentar el nivel de tu oficio inicial, tu personaje comienza en el nivel 1 del nuevo oficio. Al cambiar el oficio de tu personaje, mantienes todas las habilidades aprendidas, mejoras de atributos y competencia con el equipo de tu oficio inicial. La única excepción a esto es el atributo AGI, donde tu personaje solo obtiene la bonificación más alta entre los dos oficios. Sin embargo, tu personaje solo puede tener un máximo de 10 niveles en total entre ambos oficios. En consecuencia, la flexibilidad de cambiar de oficio es a costa de no ser capaz de convertirse en experto en ninguno.
\vspace{1cm}
%
\example{Cambio de Oficio}{
Después de luchar en el Santuario del Viento y de llegar a la cima, Bartz y su grupo se dan cuenta de que el Cristal del Viento ya ha sido destruido. De todas formas, el DJ otorga al grupo una subida de nivel y todos los miembros del grupo excepto Bartz pasan del nivel 2 al 3. Bartz recoge uno de los fragmentos del cristal y repentinamente siente un torrente de energía que le otorga poderes mágicos. En lugar de subir de nivel, cambia su oficio por el de Mago Negro, empezando en el nivel 1. Sin embargo, mantiene sus atributos y la capacidad de equipar espadas y armaduras de su antiguo oficio de Guerrero. Además, sus atributos se incrementan como se indica en la descripción del oficio de Mago Negro Nivel 1. La única excepción es el atributo AGI, que conserva la de su antiguo oficio ya que es más alto. Bartz también aprende los hechizos "Piro", "Hielo" y "Electro" además de las técnicas "Carga" y "Golpe fuerte" que ya sabe.
}
%
\clearpage
%
\ofjob{Dragoon}
{
	\ofquote{"Confident bastard, aren't you?"\\}{Kain}\\\\
	\includegraphics[width=\columnwidth]{./art/jobs/dragoon.jpg}\ofrow
	\accf{Dragoons} are masters of aerial combat, who strike their enemies with devastating attacks from the sky.
	They prefer spears as their weapon and have an affinity for the fire element. 
	Even though they are humanoid, it is said that Dragoons have the soul of a dragon.
}
{Spear}{Heavy Armor}{
	Level 1: & HP +23 & MP~+16 & AGI~+2,& STR~+1 \\
	Level 2: & HP~+10  & MP~+5 & STR~+1 & RES~+1 \\
	Level 3: & \multicolumn{3}{l}{Archetype Attribute Bonus} \\
	Level 4: & HP~+5  & MP~+10 & DEF~+2 &  	  \\
	Level 5: & HP~+10 & MP~+10 & STR~+1 & 		  \\ 
	Level 6: & HP~+10 & MP~+10 & RES~+1 		  \\
	Level 7: & HP~+10 & MP~+5  & STR~+1 & 	DEF~+1	  \\ 
	Level 8: & HP~+10 & MP~+10 & RES~+1 & 	 	  \\ 
	Level 9: & HP~+10  & MP~+10 & STR~+1 \\ 
	Level 10:& HP~+10  & RES~+1 & DEF~+2 
}{
	\ofjobtech{Jump}{3}{1r}{Single}{3u}{When you begin using this Tech, you jump 3u up into the air. After the cast time is up, you leap onto the target and make an Attack on him.}{}{1}\ofabilitygap
	\ofjobtech{Lancet}{3}{0r}{Single}{5u}{You reduce the target's HP and MP by 1d and increase your HP and MP by the same amount.}{}{2}\ofabilitygap
	\ofjobtech{Double Jump}{8}{1r}{Single}{3u}{When you begin using this tech, you jump 3u up into the air. After the cast time is up, you leap onto the target and make an Attack on him. You can then leap to another location within 3u. If you land on another enemy you can make an Attack on him too.}{}{6}\ofabilitygap
	\ofjobtech{Roar}{7}{0r}{5u}{Self}{All enemies in the target area make a DC 9 check and suffer Immobile for 1 round upon failure.}{\immobile}{8}\ofabilitygap
	\ofjobtech{Highwind}{24}{1r}{Single}{Self}{For the next 3 rounds, you stay up to 3u in the air from where you can move your usual distance and perform one of the following 2 actions without additional MP cost or cast time on each turn:\\ 
		\acc{Lance Barrage:} make an Attack against a target within 10u. If you hit, you score a Critical Hit.\\
		\acc{Fire Blast:} choose a target within 10u. He and all enemies within 2u of him suffer 4d fire damage.
	}{\fire}{10}
}{
	\ofarchetypet{Dragon Knight}
	{HP~+8 & MP~+12 & STR~+1 & RES~+2}
	{\ofarchetypetecha{Fire Breath}{7}{0r}{3u (front)}{Self}{You deal 2d fire damage to everyone in the target area.}{\fire}}
	{\ofarchetypepassive{Flametongue}{You gain permanent Resilience against fire damage. Furthermore, whenever you deal physical damage to an enemy, you can choose to let the damage dealt be of magical and fire type instead.}}
	{\ofarchetypereaction{Dragonheart}{Whenever you deal or receive fire damage, you gain EnSTR until the end of your next turn.}}
	{\ofarchetypetechb{Dragon Dive}{16}{1r}{3u}{7u}{When you begin using this Tech, you jump 3u up into the air. After the cast time is up you leap onto the target and deal 4d fire damage to everyone in the target area except yourself. Also, you create Hot Field in the target area that lasts for 3 rounds but does not affect you.}{\fire}}
}{
	\ofarchetypet{Valkyrie}
	{HP~+13 & MP~+7 & STR~+2 & DEF~+1}
	{\ofarchetypetecha{Full Thrust}{6}{0r}{5u (line)}{Self}{You dash forward in an up to 5u long line. Make an Attack on everyone in the way by making one damage roll that is applied to all targets that fail to evade.}{}}
	{\ofarchetypepassive{Duelist}{As long as you are in combat within 3u of one enemy and there is noone else within 3u of you, the STR bonus added to your Attacks and Abilities is doubled.}}
	{\ofarchetypereaction{Arm's Length}{Whenever an enemy walks within 2u of you, he has to make a DC~7 check and upon failure he cannot move any further towards you on this turn.}}
	{\ofarchetypetechb{Revenge}{12}{0r}{Single}{Weapon}{Make an Attack on an enemy that has damaged you since your last turn. On hit, you inflict the damage that he dealt to you before instead of your usual damage.}{}}
}
\thispagestyle{empty}
\subsection*{\huge Guerrero}
\vspace{0.3cm}
"Me da igual…". \\
\indent -- Squall 
\vspace{0.3cm} \\
Los Guerreros son especialistas en combate cuerpo a cuerpo debido a su poderosa capacidad física tanto en ataque como en defensa. Son expertos en el uso de espadas y armaduras, lo que les permite ser aún más peligrosos y resistentes. En su búsqueda de oponentes más fuertes, los Guerreros más experimentados saben que siempre hay un pez más gordo.
\vfill
\battrt{ \textbf{Nivel 1:} & PV~+25 & PM~+12 & AGI~+3 & FUE~+1 \\
 \textbf{Nivel 2:} & PV~+10 & PM~+5 & FUE~+1 & DEF~+1 \\
 \textbf{Nivel 3:} & PV~+10 & PM~+10 & RES~+1 &        \\
}{Espada}{Armadura Liviana, Armadura Pesada}
\vfill
\atypet{Caballero Oscuro} { \textbf{Nivel 4:} & PV~+5 & PM~+10 & FUE~+1 & RES~+1 \\ 
 \textbf{Nivel 5:} & PV~+10 & PM~+5 & DEF~+2 & 		  \\ 
 \textbf{Nivel 6:} & PV~+5 & PM~+10 & FUE~+2 &		  \\ 
 \textbf{Nivel 7:} & PV~+10 & PM~+5 & FUE~+1 & RES~+1 \\ 
 \textbf{Nivel 8:} & PV~+5 & PM~+5 & DEF~+1 & RES~+2 \\ 
 \textbf{Nivel 9:} & PV~+10 & PM~+5 & FUE~+1 & DEF~+1 \\ 
 \textbf{Nivel 10:}& PV~+10 & PM~+10 & RES~+1 &		  \\ 
} {Sacrificio} { Cuando hagas un \hyperlink{action}{Ataque} con éxito sobre un enemigo, puedes infligir daño \hyperlink{action}{Oscuro} equivalente a la mitad del daño original a ti y a todos los enemigos que se encuentren a ~3u. } {Precio de Sangre} { Siempre que un enemigo o un aliado (dispuesto) que se encuentre a 5u consuma PM, puedes obligarlo a que consuma sus PV en lugar de sus PM si tiene los suficientes PV para hacerlo. Luego, la mitad de ese valor se recupera a tus PV. }
\vfill
\atypet{Luchador} { \textbf{Nivel 4:} & PV~+10 & PM~+5 & FUE~+2 &        \\ 
 \textbf{Nivel 5:} & PV~+5 & PM~+10 & FUE~+1 & DEF~+1 \\ 
 \textbf{Nivel 6:} & PV~+10 & PM~+5 & DEF~+1 & RES~+1 \\ 
 \textbf{Nivel 7:} & PV~+5 & PM~+5 & FUE~+2 &        \\ 
 \textbf{Nivel 8:} & PV~+10 & PM~+5 & RES~+2 &        \\ 
 \textbf{Nivel 9:} & PV~+5 & PM~+10 & FUE~+1 & DEF~+1 \\ 
 \textbf{Nivel 10:}& PV~+10 & PM~+5 & DEF~+2 &        \\ 
} {Adrenalina} { Siempre que reduzcas los PV de un enemigo a 0, obtienes inmediatamente un turno extra. } {Punto Ciego} { Siempre que un enemigo en un radio de 1u inflija daño a un aliado o reciba daño de un aliado, inmediatamente puedes hacer un \hyperlink{action}{Ataque} sobre él. }
\pagebreak \\
\noindent {\Large\color{accent}\bf \uline{Habilidades\phantom{y}\hfill}}\\\\
\techt{Carga}{3}{0t}{Único}{Arma}{ Realiza un \hyperlink{action}{Ataque} contra el objetivo. Si lo golpeas, lo haces retroceder 1u además del daño infligido. }{}{1} \techt{Golpe Fuerte}{5}{0t}{Único}{Arma}{ Realiza un \hyperlink{action}{Ataque} en el que el objetivo tiene \hyperlink{check}{Ventaja} en la tirada de evasión. Si lo golpeas, haces \hyperlink{action}{Daño Crítico}. }{}{2} \techt{Rompearmadura}{10}{0t}{Único}{Arma}{ Realiza un \hyperlink{action}{Ataque} contra el objetivo. Si lo golpeas, el objetivo sufre \hyperlink{status}{disDEF} y \hyperlink{status}{disRES} por 3 turnos además del daño infligido. }{\dedef \deres}{3} \techt{Rompehuesos}{8}{0t}{Único}{Arma}{ Realiza un \hyperlink{action}{Ataque} contra el objetivo. Si lo golpeas, el objetivo queda \hyperlink{status}{Inmóvil} por 1 turno además del daño infligido. }{\immobile}{5} \techt{Enfocar}{6}{1t}{Objetivo}{Tú}{ Por los próximos 3 turnos, cada vez que realices un \hyperlink{action}{Ataque} sobre un enemigo, este tiene desventaja en la tirada de evasión. }{}{6} \techt{Valentía}{10}{1t}{2u}{Tú}{ Tú y todos los aliados que se encuentren en el área de efecto reciben \hyperlink{status}{aumFUE} y \hyperlink{status}{aumMAG} por 3 turnos. }{\enstr \enmag}{7} \techt{Vendaval}{8}{1t}{5u (línea)}{Tú}{ Realiza un \hyperlink{action}{Ataque} contra todos los que se encuentren en el área de efecto haciendo una sola tirada de daño que aplica a todos los afectados y que fallen la tirada de evasión. El daño infligido es de \hyperlink{type}{Viento}. }{\wind}{8} \techt{Andanada}{20}{1t}{5u}{Tú}{ Realiza un \hyperlink{action}{Ataque} contra todos los enemigos en el área de efecto haciendo una sola tirada de daño que aplica a todos los afectados y que fallen la tirada de evasión. Además, obtienes \hyperlink{status}{Reflejos} hasta el inicio de tu próximo turno. }{\blink}{9} \techt{Omnilátigo}{30}{0t}{Único}{Arma}{ Realiza 3 \hyperlink{action}{Ataques} separados contra el objetivo. Cada vez que el objetivo obtenga 4 o menos en la tirada de evasión, haces \hyperlink{action}{Daño Crítico}. }{}{10}
\pagebreak
\thispagestyle{empty}
\thispagestyle{empty}
\subsection*{\huge Summoner}
\vspace{0.3cm}
"I don’t like your plan. It sucks." \\
\indent -- Yuna 
\vspace{0.3cm} \\
Summoners are powerful spellcasters that can summon magical beasts to aid them in combat. 
They create a strong bond to their summon allowing the summoner to control their incredible powers to his
will. 
While the summoners themselves focus on using defensive magic, their summons can wreak havoc unlike any human being.
\vfill
\battrt{
	\textbf{Level 1:} & HP +16 & MP~+19 & AGI~+2 & MAG~+1 \\ 
	\textbf{Level 2:} & HP~+5  & MP~+10 & RES~+1 & STR~+1 \\ 
	\textbf{Level 3:} & HP~+10 & MP~+10 & MAG~+1 &         
}{Staff}{Robe}
\vfill
\atypet{Devout}
{	
	\textbf{Level 4:} & HP~+5  & MP~+10 & RES~+1 & DEF~+1 \\  
	\textbf{Level 5:} & HP~+10 & MP~+10 & MAG~+1 &        \\  
	\textbf{Level 6:} & HP~+5  & MP~+10 & MAG~+1 & RES~+1 \\  
	\textbf{Level 7:} & HP~+5  & MP~+10 & RES~+2 &        \\  
	\textbf{Level 8:} & HP~+10 & MP~+10 & DEF~+1 &		  \\  
	\textbf{Level 9:} & HP~+5  & MP~+10 & MAG~+1 & RES~+1 \\  
	\textbf{Level 10:}& HP~+10 & MP~+10 & MAG~+1 &        \\  
}
{Soulbind}
{	
	On your turn, your currently active summon can cast a spell where he can spend your MP in addition to his own and the spell's cast time is reduced by 1 round.
	You have to skip your own turn to use this effect.
}
{Sacrifice}
{	
	Whenever your currently active summon would receive any damage, you can choose to reduce your own HP by the same amount instead.
}
\vfill
\atypet{Evoker}
{		
	\textbf{Level 4:} & HP~+10 & MP~+5  & MAG~+1 & DEF~+1 \\  
	\textbf{Level 5:} & HP~+10 & MP~+10 & RES~+1 &		  \\  
	\textbf{Level 6:} & HP~+5  & MP~+10 & MAG~+1 & RES~+1 \\  
	\textbf{Level 7:} & HP~+10 & MP~+5  & DEF~+1 & RES~+1 \\  
	\textbf{Level 8:} & HP~+5  & MP~+10 & MAG~+2 &	      \\  
	\textbf{Level 9:} & HP~+5  & MP~+10 & RES~+1 & MAG~+1 \\  
	\textbf{Level 10:}& HP~+10 & MP~+10 & RES~+1 &		  \\  
}
{Channel}
{	
	On your turn, you can choose to cast a spell known by your currently active summon and the spell's cast time is reduced by 1 round. 
	The summon has to skip his turn to you use this effect.
}
{Lifesiphon}
{	
	Whenever you would receive any damage, you can choose to reduce the HP of your currently active summon by the same amount instead.
}
\pagebreak \\
\noindent {\Large\color{accent}\bf \uline{Abilities\phantom{y}\hfill}}\\\\
\spellt{Summon}{8}{3r}{Single}{Self}
{
	You summon a creature that acts with you on your turn, following your command.
	The summon is dismissed when you or the summon suffers \hyperlink{status}{KO}, but you can also dismiss it whenever you want.
	Once dismissed, you cannot summon the same creature again on the same day.
	All creatures that you can summon at different Levels are shown on the next page.
}{}{1}
\spellt{Pray}{5}{1r}{1u}{Self}{
	Everyone in the target area regains 1d HP. 
}{}{2}
\spellt{Image}{10}{1r}{1u}{3u}{
	The target gains \hyperlink{status}{Blink} for 3 rounds.
}{\blink}{4}
\spellt{Toad}{16}{1r}{Single}{3u}{
	The target makes a DC 8 check and is turned into a toad upon failure for 3 rounds or until he receives any damage.
	While being a toad, the target cannot talk or take any action and can only move 1u per turn.
}{}{6}
\spellt{Dispel}{20}{1r}{Single}{3u}{
	All \hyperlink{type}{Resiliences} and \hyperlink{status}{Immunities} of the target are removed for 3 rounds.
	Also, all beneficial \hyperlink{status}{Status Effects} that are active on the target when this spell takes effect are completely removed as well.
}{}{8}
\spellt{Twin Summon}{28}{5r}{Single}{Self}{
	You summon two different creatures that both follow your command and act with you on your turn.
	The summons are dismissed when you or they suffer \hyperlink{status}{KO}, but you can also dismiss them whenever you want.
	Once dismissed, you cannot summon the same creature again on the same day.
	All creatures that you can summon at different Levels are shown on the next page. 
}{}{10}
\vspace{5cm}
\pagebreak
\onecolumn
\noindent{\LARGE\color{accent}\bf \uline{Summons\hfill} \\} \\
\thispagestyle{empty}

\begin{multicols}{2}
\friendly{Carbuncle}{1}{\includegraphics[width=0.18\textwidth]{./art/monsters/carbuncle.png}}
{
	HP: & \hfill 20 & MP: & \hfill 36\\
	STR: & \hfill 1 & DEF: & \hfill 0 \\
	MAG: & \hfill 2 & RES: & \hfill 2 \\
	AGI: & \hfill 3 & Size: & \hfill S\\
}
{
	\textbf{Tackle}: 1d DMG\phantom{y} 
	
	\mspell{Reflect}{12}{1r}{Single}{3u}{The target gains a shield that reflects the next spell that targets them back to its caster.}{}		
}
\vspace{0.5cm} 
\friendly{Ifrit}{3}{\includegraphics[width=0.23\textwidth]{./art/monsters/ifrit.png}}
{
	HP: & \hfill 50 & MP: & \hfill 36\\
	STR: & \hfill 2 & DEF: & \hfill 3 \\
	MAG: & \hfill 1 & RES: & \hfill 0 \\
	AGI: & \hfill 3 & Size: & \hfill M\\
}
{
	\textbf{Claw}: 2d DMG \\
	\textbf{Resilience}:\fire \hfill \textbf{Weakness:}\ice
	
	\mspell{Fire}{4}{1r}{Single}{3u}{You deal 2d \hyperlink{fire}{fire} damage to the target.}{\fire}	
	\mtech{Hellfire}{12}{1r}{2u}{Self}{You deal 4d \hyperlink{type}{fire} damage to everyone in the target area except yourself.}{\fire}	
}
\vspace{0.5cm} 
\friendly{Shiva}{5}{\includegraphics[width=0.18\textwidth]{./art/monsters/shiva.png}}
{
	HP: & \hfill 60 & MP: & \hfill 80\\
	STR: & \hfill 1 & DEF: & \hfill 1 \\
	MAG: & \hfill 5 & RES: & \hfill 4 \\
	AGI: & \hfill 3 & Size: & \hfill M\\
}
{
	\textbf{Icicle}: 2d DMG, 3u Range \\
	\textbf{Resilience}:\ice \hspace*{\fill} \textbf{Weakness:}\fire
	
	\mspell{Deprotect}{5}{1r}{Single}{3u}{The target suffers \hyperlink{status}{DeDEF} for 3 rounds.}{\dedef}
	\mspell{Deshell}{5}{1r}{Single}{3u}{The target suffers \hyperlink{status}{DeRES} for 3 rounds.}{\deres}
	\mtech{Ice Wall}{10}{1r}{3u (line)}{3u}{
	You create a 3u tall and wide wall of ice that blocks the path for 5 rounds.
	The wall breaks down after 3 rounds or upon suffering a total of 30 damage.
	}{}
	\mspell{Diamond Dust}{20}{1r}{3u (front)}{Self}{
	All enemies in the target area suffer 6d \hyperlink{type}{ice} damage and \hyperlink{status}{Immobile} for 1 round.	
	}{\ice\immobile}		
}
\friendly{Phoenix}{7}{\includegraphics[width=0.2\textwidth]{./art/monsters/phoenix.png}}
{
	HP: & \hfill 70 & MP: & \hfill 90\\
	STR: & \hfill 0 & DEF: & \hfill 2 \\
	MAG: & \hfill 6 & RES: & \hfill 8 \\
	AGI: & \hfill 2 & Size: & \hfill M\\
}
{
	\textbf{Beak}: 1d DMG  \\ 
	\textbf{Immune}: \hyperlink{status}{All Status Effects} \hfill \textbf{Resilience:}\fire\holy
	
	\mspell{Protect}{5}{1r}{Single}{3u}{The target gains \hyperlink{status}{EnDEF} for 3 rounds.}{\enndef}
	\mspell{Shell}{5}{1r}{Single}{3u}{The target gains \hyperlink{status}{EnRES} for 3 rounds.}{\enres}
	\mspell{Curaga}{18}{1r}{1u}{3u}{Everyone in the target area regains 6d HP.}{}
	\mspell{Full-Life}{28}{3r}{Single}{3u}{Remove \hyperlink{status}{KO} from the target and fully heal his HP.}{\ko}	
	
}
\vspace{0.5cm} 
\friendly{Bahamut}{9}{\includegraphics[width=0.23\textwidth]{./art/monsters/bahamut.png}}
{
	HP: & \hfill 100 & MP: & \hfill 140\\
	STR: & \hfill 8 & DEF: & \hfill 6 \\
	MAG: & \hfill 7 & RES: & \hfill 4 \\
	AGI: & \hfill 4 & Size: & \hfill L\\
}
{
	\textbf{Claw}: 3d DMG, 2u Range \\
	\textbf{Immune}: \hyperlink{status}{All Status Effects} \hfill \textbf{Resilience:}\dark 

	\mtech{Obliterating Breath}{20}{1r}{3u (front)}{3u}{
	Everyone in the target area makes a DC 8 check and suffers 4d damage as well as \hyperlink{status}{Poison} and \hyperlink{status}{Blind} for 3 rounds upon failure.
	}{\poison \blind}
	\mspell{Banish}{30}{1r}{Single}{3u}{
	The target makes a DC 8 check and upon failure he is banished into another dimension and thus removed from the battlefield for 3 rounds.
	}{}
	\mspell{Megaflare}{40}{3r}{Single}{8u}{
		You deal 10d+20 \hyperlink{type}{fire} damage to the target.
	}{\fire}
	\mreaction{Final Attack}{If you are about to fall to 0 HP you may use one of your abilities without cost or cast time before falling \hyperlink{status}{KO}.}	
}
\end{multicols}
\twocolumn
\ofjob{Thief}
{
	\ofquote{"I PREFER the term ”treasure hunter!"\\}{Locke}\\\\
	\includegraphics[width=\columnwidth]{./art/jobs/thief.jpg}\ofrow
	\accf{Thieves} are mobile melee fighters, who can quickly traverse the battlefield and are difficult
	to hit with physical attacks. 
	They excel at "borrowing" items and money from enemies and have a heightened sense for worthwhile business. 
%	One would be advised to be careful when dealing with a Thief, they always have one more trick up their
%	sleeve than you would expect.
}
{Dagger}{Light Armor}{
	Level 1: & HP +20 & MP~+19 & AGI~+4 \\
	Level 2: & HP~+5  & MP~+10  & STR~+1 & DEF~+1 \\
	Level 3: & \multicolumn{3}{l}{Archetype Attribute Bonus} \\
	Level 4: & HP~+10 & MP~+5  & STR~+1 & DEF~+1 \\
	Level 5: & HP~+10 & MP~+10 & STR~+1 &        \\ 
	Level 6: & HP~+5  & MP~+5  & DEF~+2 & RES~+1 \\ 
	Level 7: & HP~+10 & MP~+10 & STR~+1 &        \\ 
	Level 8: & HP~+10 & MP~+5  & RES~+2 &        \\ 
	Level 9: & HP~+5  & MP~+10 & STR~+2 &        \\ 
	Level 10:& HP~+10 & MP~+10  & DEF~+1 
}{
	\ofjobtech{Steal}{4}{0r}{Single}{Weapon}{Make a DC 7 check and "borrow" something from the target if you succeed. Roll 1d and the you get 2d times 10G on 1 or 2, a Potion on a 3, a Remedy on a 4, an Ether on a 5 and a Phoenix Down on a 6. The item may also be determined in any other way by the GM.}{}{1} \ofabilitygap
	\ofjobtech{Flee}{3}{1r}{3u}{Self}{You and all allies in the target area can move twice as fast when running away from enemies for 3 rounds.}{}{2} \ofabilitygap
	\ofjobtech{Poison Dart}{6}{0r}{Single}{4u}{The target makes a DC~7 check and suffers 1d damage and one of the following Status Effects of your choice for 3 rounds upon failure: Poison, Immobile, Sleep}{}{4}\ofabilitygap
	\ofjobtech{Vanish}{8}{0r}{Single}{Weapon}{You become invisible for up to  5 rounds or until you take an action. While invisible, you gain Blink and have Advantage on all stealing related checks. Also, if you hit an Attack while invisible, you always score a Critical Hit.}{\blink}{6} \ofabilitygap
	\ofjobtech{Mirror Image}{23}{1r}{Single}{1u}{You create an exact copy of yourself. The clone lasts for up to 3 rounds and acts with you on your turn, following your command. The clone can use the same abilities except this one. You cannot create a clone while a previous one is still active.}{}{10}
}{
	\ofarchetypet{Ninja}
	{HP~+14 & MP~+11 & STR~+2}
	{\ofarchetypetecha{Throw}{4}{0r}{Single}{5u}{You throw a piece of equipment from your inventory on the target and deal an amount of damage depending on its equipment rank. The damage dealt is 1d for Beginner, 2d for Advanced and 3d for Expert level equipment. You can collect all thrown objects at the end of the battle.}{}}
	{\ofarchetypepassive{First Strike}{When an ally chooses you to take the next turn, you can immediately take it instead of waiting for a turn of the opposing party.}}
	{\ofarchetypereaction{Counter Attack}{When an enemy hits you with an Attack, you can immediately make an Attack on him if he is within range.}}
	{\ofarchetypetechb{Assassinate}{14}{0r}{Single}{3u}{Move to the target and make an Attack on him. If you hit, he makes a DC~7 check and suffers KO upon failure.}{\ko}}
}{
	\ofarchetypet{Treasure Hunter}
	{HP~+10 & MP~+20 & DEF~+1 & RES~+1}
	{\ofarchetypetecha{Quick Pockets}{6}{0r}{Single}{Self}{Make an Attack after which you can immediately use an Item.}{}}
	{\ofarchetypepassive{Gilionaire}{Whenever you deal damage to an enemy, you also receive an amount of Gil equal to the damage dealt.}}
	{\ofarchetypereaction{Counter Steal}{Whenever you evade an Attack by an enemy, you can immediately use "Steal" on him without any cost.}}
	{\ofarchetypetechb{Gil Toss}{8}{0r}{Single}{5u}{Throw an amount of Gil on the target up to maximum of 100G. The target suffers 1d damage for every 20G thrown.}{}}
}
\ofjob{White Mage}
{
	\ofquote{"Hey, that’s Cloud’s line! ’It’s too dangerous, I can’t get you involved...' Blah blah blah."\\}{Aerith}\\\\
	\includegraphics[width=\columnwidth]{./art/jobs/whitemage.jpg}\ofrow
	\accf{White Mages} are experts of defensive magic and boast a variety of recovery and protective spells.
	While mediocre in physical combat, they also feature incredible resistance against magic. 
%	Where others will succumb to the God of Death, a skilled White Mage will face him and say: "Not today".	
}
{Staff}{Robe}{
	Level 1: & HP~+19 & MP~+25 & AGI~+2 & STR~+1 \\
	Level 2: & HP~+5  & MP~+10 & MAG~+1 & RES~+1 \\
	Level 3: & \multicolumn{3}{l}{Archetype Attribute Bonus} \\
	Level 4: & HP~+10 & MP~+5 & MAG~+1 & DEF~+1	  \\
	Level 5: & HP~+5  & MP~+10 & RES~+1 &	STR~+1  \\ 
	Level 6: & HP~+5  & MP~+5 & MAG~+2 &        \\ 
	Level 7: & HP~+10  & MP~+10 & RES~+1 & DEF~+1 \\ 
	Level 8: & HP~+5 & MP~+5  & MAG~+2 & DEF~+1 \\ 
	Level 9: & HP~+5  & MP~+10 & RES~+1 &	MAG~+1   \\ 
	Level 10:& HP~+10 & MP~+10 & RES~+1 &	        
}{	
	\ofjobspell{Cure}{4}{0r}{Single}{3u}{The target regains 2d HP.}{}{1}\ofabilitygap
	\ofjobspell{Drain}{6}{0r}{Single}{3u}{Deal 1d damage to the target and increase your own HP by the total amount of damage dealt.}{}{2}\ofabilitygap
	\ofjobspell{Esuna}{6}{0r}{Single}{5u}{You remove all negative Status Effects except KO from the target.}{}{4}\ofabilitygap
	\ofjobspell{Curaga}{14}{1r}{2u}{5u}{Everyone in the target area regains 6d HP.}{}{6}\ofabilitygap
	\ofjobspell{Clear}{6}{0r}{5u}{50u}{You remove one active Field Effect within range.}{}{6}\ofabilitygap
	\ofjobspell{Holy}{21}{2r}{Single}{12u}{You deal 6d+45 holy damage to the target.}{\holy}{8}\ofabilitygap
	\ofjobspell{Auto-Life}{28}{2r}{Single}{3u}{You summon a guardian angel that watches over the target. The next time he falls KO, he is instantly revived with 1 HP. This effect does not stack and if not activated, it expires when the target goes to sleep.}{\ko}{10}
}{
	\ofarchetypet{Sage}
	{HP~+11 & MP~+9 & MAG~+2 & STR~+1}
	{\ofarchetypespella{Sleep}{6}{0r}{Single}{5u}{The target makes a DC 8 check and suffers Sleep for 3 rounds upon failure.}{\sleep}
	\vspace*{0.1cm}\\ \ofarchetypespella{Silence}{6}{0r}{Single}{5u}{The target makes a DC 8 check and suffers Silence for 3 rounds upon failure.}{\silence}}
	{\ofarchetypepassive{Ancient Wisdom}{Whenever you inflict on or more Status Effects on a target, you can also inflict DeDEF or DeRES on him for 3 rounds.}}
	{\ofarchetypereaction{Absorb MP}{When you are the target of an enemy ability, increase your MP by half the amount that the caster spent on it.}}
	{\ofarchetypespellb{Curse}{14}{1r}{Single}{5u}{The target makes a DC~9 check and upon failure he suffers 4d damage as well as Poison and Zombie for 3 rounds.}{}}
}{
	\ofarchetypet{Medic}
	{HP~+7 & MP~+13 & RES~+2 & DEF~+1}
	{\ofarchetypespella{Protect}{4}{0r}{Single}{5u}{The target gains EnDEF for 3 rounds.}{\enndef} \vspace*{0.1cm}\\ \ofarchetypespella{Shell}{4}{0r}{Single}{5u}{The target gains EnRES for 3 rounds.}{\enres}}
	{\ofarchetypepassive{Doctor's Code}{Whenever you use Magic on an ally within 1u, you can also immediately use an Item on him in addition.}}
	{\ofarchetypereaction{No Collateral}{Whenever you would be affected by a spell or tech that you are not the primary target of, you can choose that you and all other secondary targets are unaffected.}}
	{\ofarchetypespellb{Full-Life}{22}{2r}{Single}{5u}{You remove the KO status from the target and fully restore his HP.}{\ko}}
}
\thispagestyle{empty}
\subsection*{\huge Black Mage}
\vspace{0.3cm}
"You sure are a keen observer of the obvious, kupo!" \\
\indent -- Montblanc 
\vspace{0.3cm} \\
Black magic is a pathway to many abilities some consider to be unnatural. 
Black Mages are fragile in physical combat, but can wipe out multiple enemies from great distances and inflict nasty status effects. 
They can thus assert great control over the battlefield and are difficult to ignore for enemies. \\
\vfill
\battrt
{
	\textbf{Level 1:} & HP~+17 & MP~+21 & AGI~+2 & MAG~+1  \\
	\textbf{Level 2:} & HP~+5  & MP~+10 & STR~+1 & RES~+1  \\
	\textbf{Level 3:} & HP~+10 & MP~+10 & MAG~+1 &         
}
{Staff}
{Robe}
\vfill
\atypet{Arcanist}
{
	\textbf{Level 4:} & HP~+5  & MP~+5  & MAG~+2 & DEF~+1 \\ 
	\textbf{Level 5:} & HP~+5  & MP~+10 & RES~+1 & MAG~+1 \\ 
	\textbf{Level 6:} & HP~+5  & MP~+10 & RES~+1 & DEF~+1 \\
	\textbf{Level 7:} & HP~+10 & MP~+10 & MAG~+1 &  	  \\
	\textbf{Level 8:} & HP~+5  & MP~+10 & MAG~+1 & DEF~+1 \\
	\textbf{Level 9:} & HP~+10 & MP~+5  & RES~+1 & MAG~+1 \\
	\textbf{Level 10:}& HP~+5  & MP~+10 & RES~+2 &		  
}
{Magic Boost}
{	
	Whenever you cast \hyperlink{action}{Magic} that targets a single entity, you can choose to also target everyone within 1u of him. The damage dealt to secondary targets is halved.
}
{Critical Vanish}
{	
	Whenever you have more than 1 HP and an \hyperlink{action}{Attack} would reduce you to 0 HP, you remain at 1 HP and gain \hyperlink{status}{Blink} for 3 rounds or until you take an action.
}
\vfill
\atypet{Scholar}
{	
	\textbf{Level 4:} & MP~+10 & RES~+1 & DEF~+1 & MAG~+1 \\
	\textbf{Level 5:} & HP~+10 & MP~+10 & MAG~+1 		  \\
	\textbf{Level 6:} & HP~+5  & MP~+10 & RES~+1 & MAG~+1 \\
	\textbf{Level 7:} & HP~+5  & MP~+10 & MAG~+2  		  \\
	\textbf{Level 8:} & HP~+5  & MP~+10 & RES~+1 & DEF~+1 \\
	\textbf{Level 9:} & HP~+5  & MP~+10 & RES~+1 & MAG~+1  \\
	\textbf{Level 10:}& HP~+10 & MP~+10 & MAG~+1		  \\
}
{Turbo MP}
{	
	Whenever you begin casting \hyperlink{action}{Magic}, you can choose to double its range by also doubling the MP cost.
}
{Return Magic}
{	
	Whenever you suffer damage caused by \hyperlink{action}{Magic}, you can cast the same spell back to its caster.
	In doing this, you have to respect the cast time and MP cost of the spell.
	If you are already casting another spell, you have to break its concentration to use this effect. 
}
\pagebreak \\
\noindent {\Large\color{accent}\bf \uline{Abilities\phantom{y}\hfill}}\\\\
\spellt{Fire}{4}{1r}{Single}{3u}
{
	You deal 2d \hyperlink{type}{fire} damage to the target.
}{\fire}{1}
\spellt{Blizzard}{4}{1r}{Single}{3u}
{
	You deal 2d \hyperlink{type}{ice} damage to the target.
}{\ice}{1}
\spellt{Thunder}{4}{1r}{Single}{3u}
{
	You deal 2d \hyperlink{type}{lightning} damage to the target.
}{\lightning}{1}
\spellt{Blind}{6}{1r}{Single}{3u}
{
	The target makes a DC 8 check and suffers \hyperlink{status}{Blind} for 3 rounds upon failure.
}{\blind}{2}
\spellt{Bio}{8}{1r}{Single}{3u}
{
	The target makes a DC 8 check and suffers 2d damage and \hyperlink{status}{Poison} for 3 rounds upon failure.
}{\poison}{3}
\spellt{Firaga}{12}{2r}{Single}{5u}
{
	You deal 6d \hyperlink{type}{fire} damage to the target. 
}{\fire}{5}
\spellt{Blizzaga}{12}{2r}{Single}{5u}
{
	You deal 6d \hyperlink{type}{ice} damage to the target. 
}{\ice}{5}
\spellt{Thundaga}{12}{2r}{Single}{5u}
{
	You deal 6d \hyperlink{type}{lightning} damage to the target. 
}{\lightning}{5}
\spellt{Rasp}{4}{1r}{Single}{5u}
{
	You reduce the target's MP by 4d.
}{}{6}
\spellt{Quake}{22}{2r}{3u}{8u}
{
	Deal 8d \hyperlink{type}{earth} damage to everyone in the target area. 
}{\earth}{7}
\spellt{Flare}{24}{3r}{Single}{5u}
{
	You deal 9d+15 \hyperlink{type}{fire} damage to the target. 
}{\fire}{8}
\spellt{Doom}{28}{1r}{Single}{5u}
{
	The target makes a DC 8 check and suffers \hyperlink{status}{KO} after 3 rounds upon failure.
}{\ko}{9}
\spellt{Ultima}{40}{3r}{3u}{5u}
{
	Deal 10d+20 \hyperlink{type}{dark} damage to all enemies in the target area. 
}{\dark}{10}
\pagebreak
\ofjob{Red Mage}
{
	\ofquote{"Oh, I’ll show you how lightning strikes."\\}{Lightning}\\\\
	\includegraphics[width=\columnwidth]{./art/jobs/redmage.jpg}\ofrow
	\accf{Red Mages} are very versatile and possess a wide variety of abilities, but can also hold their own in melee combat. 
	Although they excel in neither discipline, Red Mages are still a force to be reckoned with.
}
{Rod or Sword}{Light Armor or Robe}{
	Level 1: & HP~+20 & MP~+21 & AGI~+3 & STR +1 \\
	Level 2: & HP~+5  & MP~+10 & MAG~+1 & DEF~+1 \\
	Level 3: & \multicolumn{3}{l}{Archetype Attribute Bonus}   \\
	Level 4: & HP~+10 & MP~+5  & STR~+1 & RES~+1 \\   
	Level 5: & HP~+5  & MP~+10 & MAG~+2 & 		  \\ 
	Level 6: & HP~+5 & MP~+10  & STR~+1 &	MAG~+1 \\ 
	Level 7: & HP~+10  & MP~+10 & DEF~+1 \\ 
	Level 8: & HP~+10 & MP~+5  & STR~+1 & MAG~+1 \\ 
	Level 9: & HP~+5  & MP~+10 & RES~+2 \\ 
	Level 10: & HP~+10 & MP~+10 & STR~+1 		  
}{	
	\ofjobspell{Cure}{4}{0r}{Single}{3u}{The target regains 2d HP.}{}{1}\ofabilitygap
	\ofjobspell{Fire}{4}{0r}{Single}{3u}{You deal 2d fire damage to the target.}{\fire}{2}\ofabilitygap
	\ofjobspell{Blizzard}{4}{0r}{Single}{3u}{You deal 2d ice damage to the target.}{\ice}{2}\ofabilitygap
	\ofjobspell{Thunder}{4}{0r}{Single}{3u}{You deal 2d lightning damage to the target.}{\lightning}{2}\ofabilitygap
	\ofjobspell{Blind}{6}{0r}{Single}{5u}{The target makes a DC~8 check and suffers Blind for 3 rounds upon failure.}{\blind}{4}\ofabilitygap	
	\ofjobspell{Esuna}{6}{0r}{Single}{3u}{You remove all negative Status Effects except KO from the target.}{}{6}\ofabilitygap
	\ofjobspell{NulElement}{10}{0r}{Single}{5u}{Choose an element (e.g. fire). The target does not suffer any damage of the chosen element for 3 rounds.}{}{8}\ofabilitygap
	\ofjobspell{Dualcast}{4}{0r}{Single}{Self}{You begin casting and concentrating on two spells of your choice simultaneously, but need to spend the necessary MP for both.}{}{10}
}{
	\ofarchetypet{Ravager}
	{HP~+6 & MP~+14 & MAG~+2 & RES~+1}
	{\ofarchetypespella{Poison}{6}{0r}{Single}{5u}{The target makes a DC~8 check and suffers Poison for 3 rounds upon failure.}{\poison}}
	{\ofarchetypepassive{Stagger}{Whenever you inflict one or more Status Effects on an enemy, he additionally suffers an amount of magical damage equal to your MAG.}}
	{\ofarchetypereaction{Swiftcast}{When you suffer damage from an enemy, you can immediately use an ability on him if he is within range.}}
	{\ofarchetypespellb{Imperil}{8}{0r}{Single}{5u}{The target suffers DeDEF and DeRES for 3 rounds.}{\dedef \deres} \ofabilitygap \ofarchetypespellb{Wall}{8}{0r}{Single}{5u}{The target gains EnDEF and EnRES for 3 rounds.}{\enndef \enres}}
}{
	\ofarchetypet{Spellblade}
	{HP~+14 & MP~+6 & STR~+2 & DEF~+1}
	{\ofarchetypetecha{Elemental Strike}{4}{0r}{Single}{Weapon}{Choose an element (e.g. fire) and make an Attack. If you hit, the damage is of magical type with the chosen element and you also add your MAG to the damage dealt.}{}}
	{\ofarchetypepassive{Magic Weapon}{Whenever you cast Magic, you can choose to store the spell inside your weapon. In this case, the spell's MP cost is halved. All stored spells take effect together with your next successful Attack and you can chose targets within their range including yourself. You cannot store more than two spells at once inside your weapon.}}
	{\ofarchetypereaction{Mana Shield}{Whenever your HP is reduced, you can instead choose to reduce your MP by the same amount.}}
	{\ofarchetypetechb{Phantom Blade}{5}{0r}{Single}{5u}{Make an Attack on a target within range as if you were standing next to him. This Attack cannot be evaded.}{}}
}
\thispagestyle{empty}
\subsection*{\huge Mago del Tiempo}
\vspace{0.3cm}
"¡El tiempo de jugar se acabó!" \\
\indent -- Ultimecia 
\vspace{0.3cm} \\
Los Magos del Tiempo son maestros del tiempo y el espacio, que comprenden que la imaginación es más importante que el conocimiento. Pueden manipular el flujo del tiempo y doblar el tejido de la realidad a su ventaja. Aunque los Magos del Tiempo rara vez luchan solos, pueden influir en gran medida durante una batalla con sus increíbles habilidades. \\
\vfill
\battrt{ \textbf{Nivel 1:} & PV~+17 & PM~+25 & AGI~+2 & MAG~+1 \\
 \textbf{Nivel 2:} & PV~+5 & PM~+10 & RES~+1 & FUE~+1 \\
 \textbf{Nivel 3:} & PV~+10 & PM~+10 & DEF~+1 &        \\
}{Bastón}{Túnica}
\vfill
\atypet{Ilusionista} { \textbf{Nivel 4:} & PV~+5 & PM~+10 & MAG~+2 &        \\ 
 \textbf{Nivel 5:} & PV~+10 & PM~+10 & RES~+1 &		  \\ 
 \textbf{Nivel 6:} & PV~+5 & PM~+10 & DEF~+1 & RES~+1 \\ 
 \textbf{Nivel 7:} & PV~+5 & PM~+10 & MAG~+1 & RES~+1 \\ 
 \textbf{Nivel 8:} & PV~+5 & PM~+10 & DEF~+1 & MAG~+1 \\ 
 \textbf{Nivel 9:} & PV~+5 & PM~+10 & RES~+2 &        \\ 
 \textbf{Nivel 10:}& PV~+5 & PM~+10 & MAG~+2 &		  \\ 
} {Ímpetu} { Siempre que te muevas al menos 1u en dirección a tu objetivo antes de lanzar un hechizo, el alcance del mismo aumenta 1u. } {Teletransportación Evasiva} { Siempre que seas objetivo de un \hyperlink{action}{Ataque}, puedes evadirlo teletransportándote a una ubicación que elijas a 3u de distancia. Si lo haces, no podrás utilizar ninguna habilidad por 1 turno, incluyendo esta. Tampoco puedes utilizar esta habilidad mientras estés concentrándote. }
\vfill
\atypet{Oráculo} { \textbf{Nivel 4:} & PV~+10 & PM~+5 & RES~+1 & MAG~+1 \\ 
 \textbf{Nivel 5:} & PV~+5 & PM~+10 & MAG~+1 & RES~+1 \\ 
 \textbf{Nivel 6:} & PV~+5 & PM~+10 & DEF~+2 &		   \\ 
 \textbf{Nivel 7:} & PV~+10 & PM~+10 & RES~+1 & 	      \\ 
 \textbf{Nivel 8:} & PV~+5 & PM~+10 & MAG~+2 &        \\ 
 \textbf{Nivel 9:} & PV~+5 & PM~+10 & RES~+1 & MAG~+1 \\ 
 \textbf{Nivel 10:}& PV~+5 & PM~+10 & RES~+2 &        \\ 
} {Ver Más Allá} { Puedes detectar siempre la presencia de personajes hostiles o monstruos que se encuentren en un radio de 50u. } {Karma} { Siempre tienes un dado en reserva, siendo el primero un 6. Siempre que alguien dentro de 3u de ti (incluido tú mismo) haga una tirada, puedes alterar el resultado intercambiando uno de los dados por el que tienes en reserva, si quien hizo la tirada te permite hacerlo. }
\pagebreak \\
\noindent {\Large\color{accent}\bf \uline{Habilidades\phantom{y}\hfill}}\\\\
\spellt{Gravedad}{5}{1t}{Único}{3u}{ El objetivo recibe 1d de daño mágico y solo puede moverse la mitad de la distancia habitual en su próximo turno. }{}{1} \spellt{Freno}{10}{1t}{Único}{3u}{ Por los próximos 3 turnos, el objetivo puede moverse o realizar una acción en su turno, pero no ambas. Además, todos sus tiempos de lanzamiento aumentan 1 turno. }{}{2} \spellt{Prisa}{10}{1t}{Único}{3u}{ Por los próximos 3 turnos, el objetivo puede hacer un movimiento o acción adicional en su turno y todos sus tiempos de lanzamiento se reducen 1 turno. }{}{3} \spellt{Gravedad++}{16}{2t}{2u}{5u}{ Todos los que se encuentren en el área de efecto reciben 4d de daño mágico y solo pueden moverse la mitad de la distancia habitual en su próximo turno. }{}{5} \spellt{Extender}{9}{0t}{Único}{5u}{ Elige un objetivo que esté siendo beneficiado o perjudicado por un estado alterado por un tiempo limitado. La duración de ese efecto se extiende 2 turnos. }{}{6} \spellt{Teletransportación}{14}{1t}{1u}{10u}{ Te teletransportas a una ubicación que puedas ver y que se encuentre dentro de un radio de 10u de ti. }{}{6} \spellt{Acelerar}{20}{0t}{Único}{3u}{El objetivo obtiene un turno extra inmediatamente después del tuyo. Este efecto no altera el orden de los turnos. }{}{7} \spellt{Levitar}{12}{1t}{Único}{3u}{ El objetivo puede levitar hasta 3u del piso por 3 turnos. Se mantiene su distancia habitual de movimiento. No puede recibir daño de \hyperlink{type}{Tierra} mientras esté flotando. }{}{8} \spellt{Paro}{32}{1t}{100u}{Tú}{ Detienes el tiempo para todos los que se encuentren en el área de efecto excepto para ti por hasta 1 minuto (6 turnos). Una vez que realices un \hyperlink{action}{Ataque} o lances \hyperlink{action}{Magia}, el tiempo vuelve a su curso normal nuevamente. }{}{9} \spellt{Paradoja}{45}{0t}{100u}{Tú}{ Vuelves 10 segundos en el tiempo (1 turno) dentro del área de efecto. Todas las entidades afectadas vuelven al estado (PV, PM, estados alterados) y posición del turno anterior. Sin embargo, tus PM actuales no están sujetos al efecto de "Paradoja". Además, todos los objetivos afectados no tienen ningún recuerdo de los sucesos ocurridos en el turno anterior. }{}{10}
\pagebreak

\thispagestyle{empty}
\subsection*{\huge Monk}
\vspace{0.3cm}
"Now I know why I have these stupid muscles!" \\
\indent -- Sabin 
\vspace{0.3cm} \\
Monks are adept melee fighters that posses a deadly combination of strength and technique.
While they do not have expertise in using magic, Monks can produce similarly incredible effects by tapping into their inner life force. 
A skilled monk absorbs what is useful, discards what is useless and adds what is specifically his own.
\vfill
\battrt
{
	\textbf{Level 1:} & HP +22 & MP~+10  & AGI~+4 & \\
	\textbf{Level 2:} & HP~+10 & MP~+5  & STR~+2 & \\
	\textbf{Level 3:} & HP~+10 & MP~+10 & DEF~+1 & \\
}{Claw}{Light Armor}
\vfill
\atypet{Black Belt}
{		
	\textbf{Level 4:} & HP~+10 & MP~+5  & STR~+2 &        \\ 
	\textbf{Level 5:} & HP~+5  & MP~+10 & DEF~+1 & RES~+1 \\
	\textbf{Level 6:} & HP~+10 & MP~+5  & STR~+1 & RES~+1 \\
	\textbf{Level 7:} & HP~+10 & MP~+5  & STR~+1 & DEF~+1 \\ 
	\textbf{Level 8:} & HP~+10 & MP~+5  & STR~+2 &        \\
	\textbf{Level 9:} & HP~+5  & MP~+10 & STR~+1 & RES~+1 \\ 
	\textbf{Level 10:}& HP~+10 & MP~+10 & STR~+1 &        \\ 
}
{Unscarred}
{	
	As long as your current HP is equal to your maximum HP, the STR bonus that is added to the damage dealt by your \hyperlink{action}{Attacks} is doubled.
}
{Strikeback}
{	
		Whenever you successfully evade an \hyperlink{action}{Attack} by an enemy, immediately make an \hyperlink{action}{Attack} on him.
}
\vfill
\atypet{Templar}
{	
	\textbf{Level 4:} & HP~+5  & MP~+10 & STR~+1 & RES~+1 \\ 
	\textbf{Level 5:} & HP~+10 & MP~+5  & DEF~+2 & 		  \\ 
	\textbf{Level 6:} & HP~+5  & MP~+10 & STR~+2 & 		  \\ 
	\textbf{Level 7:} & HP~+10 & MP~+5  & RES~+2 & 	      \\ 
	\textbf{Level 8:} & HP~+10 & MP~+10 & STR~+1 & 	      \\ 
	\textbf{Level 9:} & HP~+5  & MP~+10 & RES~+1 & STR~+1 \\ 
	\textbf{Level 10:}& HP~+10 & MP~+10 & STR~+1 & 		  \\ 
}
{Lifestream}
{	
	If you do not have enough MP to use an ability you can instead choose to reduce your HP by the amount of its MP cost in order to use it.
}
{Replenish MP}
{	
	Whenever you suffer \hyperlink{type}{physical} damage, increase your MP by 1d.
}
\pagebreak \\
\noindent {\Large\color{accent}\bf \uline{Abilities\phantom{y}\hfill}}\\\\
\techt{Boost}{3}{0r}{Single}{Self}
{
	You gain \hyperlink{status}{EnSTR} until the end of your next turn. 
}{\enstr}{1}
\techt{Chakra}{6}{1r}{Single}{Self}
{
	You regain 1d HP and remove all \hyperlink{status}{Status Effects} that you are currently suffering.
}{}{2}
\techt{Kick}{8}{0r}{1u}{Self}
{
	You make an \hyperlink{action}{Attack} against all enemies within 1u of you, by making one damage roll that applies to all affected targets that fail to evade. All targets that fail to evade are also knocked back by 1u. 
}{}{3}
\techt{Pummel}{9}{0r}{Single}{Weapon}
{
	You make 2 consecutive \hyperlink{action}{Attacks} against the target. 
}{}{5}
\techt{Vigilance}{6}{0r}{Single}{Self}
{
	You gain \hyperlink{status}{Blink} until the end of your next turn. 
}{\blink}{6}
\techt{Revive}{16}{3r}{Single}{1u}
{
	You remove \hyperlink{status}{KO} from the target and increase his HP by~1. 
}{\ko}{6}
\techt{Aurablast}{7}{0r}{Single}{3u}
{
	You deal 4d \hyperlink{type}{magical} damage to the target.
}{}{7}
\techt{Meteor Strike}{16}{0r}{Single}{Weapon}
{
	You slam the target into the ground dealing 7d damage.
	In doing this, you can also leap to a location of your choice within 3u.
}{}{8}
\techt{Blitz}{5}{0r}{Single}{Self}
{
	You use two different \hyperlink{action}{Techs} consecutively in the same turn.
	In doing this, you have to respect additional MP costs and cast times of both \hyperlink{action}{Techs}.
	If an enemy is affected by both \hyperlink{action}{Techs}, deal an additional 4d damage to him.
}{}{9}
\techt{Final Heaven}{24}{0r}{Single}{Weapon}
{
	You deal 6d damage to the target and knock him back by 3u.
	If he hits a wall or a similarly solid object in doing so, you deal another 4d damage to him.
}{}{10}
\pagebreak
\thispagestyle{empty}
\subsection*{\huge Sentinela}
\vspace{0.3cm}
"Permítanme destruir sus delirios de grandeza" \\
\indent -- Beatrix 
\vspace{0.3cm} \\
Los Sentinelas son maestros del combate defensivo que raramente caerán en batalla. Sus habilidades especiales les permiten no solo soportar grandes cantidades de daño, sino que también proporcionan protección a sus aliados. Un Sentinela competente es el último bastión entre el grupo y una muerte certera.
\vfill
\battrt { \textbf{Nivel 1:} & PV~+28 & PM~+13 & AGI~+3 & DEF~+1 \\
\textbf{Nivel 2:} & PV~+10 & PM~+10 & FUE~+1 & RES~+1 \\
\textbf{Nivel 3:} & PV~+10 & PM~+10 & DEF~+1 &  
}{Espada}{Armadura Pesada}
\vfill
\atypet{Defensor} { \textbf{Nivel 4:} & PV~+10 & PM~+5 & DEF~+1 & FUE~+1 \\ 
 \textbf{Nivel 5:} & PV~+10 & PM~+5 & FUE~+1 & DEF~+1 \\ 
 \textbf{Nivel 6:} & PV~+10 & PM~+10 & RES~+1 &        \\ 
 \textbf{Nivel 7:} & PV~+10 & PM~+5 & FUE~+1 & DEF~+1 \\
 \textbf{Nivel 8:} & PV~+10 & PM~+5 & RES~+1 & DEF~+1 \\
 \textbf{Nivel 9:} & PV~+10 & PM~+5 & DEF~+2 & 		\\ 
 \textbf{Nivel 10:}& PV~+10 & PM~+5 & FUE~+2 &        \\
} {Provocar} { Siempre que \hyperlink{action}{Ataques} a un enemigo con éxito, puedes intentar provocarlo. Si lo haces, el objetivo debe hacer una tirada con DC~7. Si falla, debe dirigir una acción hacia ti en su siguiente turno, si es posible. } {Bloqueo} { Cuando un enemigo que se encuentre a 1u de ti intenta alejarse de tu posición, debe hacer una tirada con DC 7. Si falla, sufre \hyperlink{status}{Inmóvil} hasta el comienzo de su próximo turno, evitando que se mueva más en este turno. }
\vfill
\atypet{Paladín} { \textbf{Level 4:} & PV~+10 & PM~+10 & DEF~+1 \\
 \textbf{Nivel 5:} & PV~+10 & PM~+5 & RES~+1 & DEF~+1 \\ 
 \textbf{Nivel 6:} & PV~+10 & PM~+5 & FUE~+2 &        \\
 \textbf{Nivel 7:} & PV~+10 & PM~+5 & RES +2 &        \\
 \textbf{Nivel 8:} & PV~+10 & PM~+5 & FUE~+1 & DEF~+1 \\
 \textbf{Nivel 9:} & PV~+10 & PM~+10 & FUE~+1 &        \\
 \textbf{Nivel 10:}& PV~+5 & PM~+10 & RES~+1 & DEF +1 \\
} {Guardia Sagrada} { Siempre que haya un aliado a 1u de ti, ambos reciben \hyperlink{status}{Reflejos}. Este beneficio no puede aplicarse a más de un aliado al mismo tiempo. } {Cubrir} { Siempre que un aliado que se encuentre a 1u de ti reciba daño \hyperlink{type}{Físico}, puedes elegir recibir la mitad del daño total en vez de que tu aliado reciba todo el daño. }
\pagebreak \\
\noindent {\Large\color{accent}\bf \uline{Habilidades\phantom{y}\hfill}}\\\\
\techt{Guardia}{3}{0t}{Único}{Tú}{ Obtienes \hyperlink{status}{aumDEF} hasta el final de tu próximo turno. }{\enndef}{1} \techt{Primeros Auxilios}{5}{0t}{Único}{1u}{ Elige a un objetivo que haya recibido daño este turno o el anterior (incluido tú). El objetivo recupera 2d de PV. }{}{2} \techt{Rompebrazo}{10}{0t}{Único}{Arma}{ Realiza un \hyperlink{action}{Ataque} sobre el objetivo. Si lo golpeas, el objetivo sufre \hyperlink{status}{disFUE} y \hyperlink{status}{disMAG} por 3 turnos además del daño infligido. }{\destr \demag}{3} \techt{Guardia Vital}{9}{1t}{Único}{Tú}{ Obtienes \hyperlink{status}{aumDEF} por 3 turnos. Cuando el efecto termine, recuperas 2d de PV. }{\enndef}{5} \spellt{Muro de Tierra}{10}{1t}{3u (línea)}{3u}{ Creas un muro de 3u de alto y ancho que bloquea el paso. El muro se rompe después de 5 turnos o tras sufrir un total de 30 puntos de daño. }{}{6} \techt{Hostigar}{8}{1t}{Único}{3u}{ El objetivo hace una tirada con DC 8. Si falla, queda \hyperlink{status}{Inmóvil} por 3 turnos. }{\immobile}{7} \spellt{Astra}{11}{1t}{Único}{3u}{ Por los próximos 3 turnos, el objetivo es \hyperlink{status}{Inmune} a todos los \hyperlink{status}{Estados Alterados}. }{}{8} \techt{Segar}{16}{0t}{Único}{Arma}{ Realiza un \hyperlink{action}{Ataque} sobre el objetivo. Si lo golpeas, el objetivo recibe como daño la diferencia entre tus PV actuales y tus PV máximos en lugar del daño normal. }{}{9} \techt{Guardia Total}{30}{1t}{Único}{Tú}{ Por los próximos 3 turnos, no recibes ningún daño. }{}{10}
\pagebreak

\thispagestyle{empty}
\subsection*{\huge Marksman}
\vspace{0.3cm}
"I play the leading man; who else?" \\
\indent -- Balthier 
\vspace{0.3cm} \\
Marksmen are experts of all kinds of ranged weapons that strike from great distance. 
Skilled Marksmen can see through their enemies, allowing them to know target's strengths and weaknesses. 
Therefore they can not only deal significant ranged damage, but also disable enemies with special techniques. 
For the Marksman there is but one rule: hunt or be hunted.
\vfill
\battrt
{
	\textbf{Level 1:} & HP +19 & MP~+12 & AGI~+2 & STR +1 \\
	\textbf{Level 2:} & HP~+5  & MP~+10 & STR~+1 & DEF~+1 \\
	\textbf{Level 3:} & HP~+10 & MP~+5  & STR~+1 & RES~+1 \\
}{Bow, Gun}{Light Armor}
\vfill
\atypet{Ranger}
{	
	\textbf{Level 4:} & HP~+10 & MP~+10 & STR~+1 &        \\
	\textbf{Level 5:} & HP~+5  & MP~+10 & DEF~+2 &        \\
	\textbf{Level 6:} & HP~+10 & MP~+10 & RES~+1 &        \\
	\textbf{Level 7:} & HP~+5  & MP~+10 & STR~+1 & RES~+1 \\
	\textbf{Level 8:} & HP~+5  & MP~+5  & RES~+1 & DEF~+2 \\
	\textbf{Level 9:} & HP~+5  & MP~+10 & RES~+1 & STR~+1 \\
	\textbf{Level 10:}& HP~+10 & MP~+5  & STR~+2 &        \\
}
{Recoil}
{	
	Whenever you make a successful \hyperlink{action}{Attack}, you can immediately move 1u even when wielding a bow.
}
{Magic Evade}
{	
	You can evade \hyperlink{action}{Magic} by passing an evasion check, the same way you evade \hyperlink{action}{Attacks}.
}
\vfill
\atypet{Sniper}
{	
	\textbf{Level 4:} & HP~+5  & MP~+5  & STR~+2 & RES~+1 \\
	\textbf{Level 5:} & HP~+10 & MP~+10 & DEF~+1 &		  \\
	\textbf{Level 6:} & HP~+5  & MP~+10 & STR~+1 & DEF~+1 \\
	\textbf{Level 7:} & HP~+5  & MP~+5  & STR~+1 & RES~+2 \\
	\textbf{Level 8:} & HP~+5  & MP~+10 & DEF~+1 & RES~+1 \\
	\textbf{Level 9:} & HP~+10 & MP~+5  & STR~+2 &        \\
	\textbf{Level 10:}& HP~+5  & MP~+5  & RES~+1 & STR~+2 \\
}
{Concentrate}
{	
	Whenever you \hyperlink{action}{Attack} an enemy, he has \hyperlink{check}{Disadvantage} on the evasion check.
}
{Auto-Item}
{	
	Whenever you suffer any damage, you can immediately use an \hyperlink{item}{Item}.
	You can only use this effect once per round.
}
\pagebreak \\
\noindent {\Large\color{accent}\bf \uline{Abilities\phantom{y}\hfill}}\\\\
	\techt{Big Shot}{3}{0r}{Single}{Weapon}
	{
		Make an \hyperlink{action}{Attack} on the target. 
		If you hit, the damage dealt ignores the target's DEF.
	}{}{1}
	\techt{Lay Trap}{4}{1r}{1u}{Self}{
		You set a trap where you are standing.
		An enemy that walks over it makes a DC 9 check and suffers 2d damage and \hyperlink{status}{Immobile} for 1 round upon failure.
		The trap disappears once it is activated.
	}{\immobile}{2}
	\spellt{Libra}{5}{0r}{Single}{3u}
	{
		You analyse the target thoroughly and know his \hyperlink{type}{Resiliences}, \hyperlink{type}{Weaknesses}, \hyperlink{type}{Immunities}, as well as his current HP and MP.
	}{}{3}
	\techt{Quick Shot}{9}{0r}{Single}{Weapon}
	{
		You make an \hyperlink{action}{Attack} after which you can immediately begin using an \hyperlink{action}{Ability} or \hyperlink{item}{Item} on the same turn.
	}{}{5}
	\techt{Pierceshot}{7}{0r}{10u (line)}{Self}
	{
		You make an \hyperlink{action}{Attack} against all targets in a line, by making one damage roll that applies to everyone that fails to evade.	
	}{}{6}
	\techt{Poison Ammo}{8}{0r}{Single}{Weapon}
	{
		Make an \hyperlink{action}{Attack} on the target. 
		If you hit, the damage dealt is \hyperlink{type}{magical} and the target makes a DC 8 check. 
		Upon failure, he suffers \hyperlink{status}{Poison} for 3 rounds.	
	}{\poison}{7}
	\techt{Target MP}{4}{1r}{Single}{Weapon}
	{
		Make an \hyperlink{action}{Attack} on the target.
		If you hit, you reduce his MP by the amount of damage dealt instead of his HP.
	}{}{8}
	\techt{Smoke Bomb}{12}{1r}{3u}{5u}
	{
		You create a smoke cloud that inhibits vision in the target area for 5 rounds.
		Everyone inside the cloud suffers \hyperlink{status}{Blind}, but gains \hyperlink{status}{Blink}.
	}{\blind\blink}{9}
	\techt{Barrage}{22}{1r}{Single}{Self}
	{
		For up to 3 rounds you make 2 consecutive \hyperlink{action}{Attacks} as your action on every turn.
		As long as this effect is active, you cannot move or take other actions.
		You can choose to end this effect at the start of every turn.
	}{}{10}
\pagebreak


