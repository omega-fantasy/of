\thispagestyle{empty}
\subsection*{\huge Sentinela}
\vspace{0.3cm}
"Permítanme destruir sus delirios de grandeza" \\
\indent -- Beatrix 
\vspace{0.3cm} \\
Los Sentinelas son maestros del combate defensivo que raramente caerán en batalla. Sus habilidades especiales les permiten no solo soportar grandes cantidades de daño, sino que también proporcionan protección a sus aliados. Un Sentinela competente es el último bastión entre el grupo y una muerte certera.
\vfill
\battrt { \textbf{Nivel 1:} & PV~+28 & PM~+13 & AGI~+3 & DEF~+1 \\
\textbf{Nivel 2:} & PV~+10 & PM~+10 & FUE~+1 & RES~+1 \\
\textbf{Nivel 3:} & PV~+10 & PM~+10 & DEF~+1 &  
}{Espada}{Armadura Pesada}
\vfill
\atypet{Defensor} { \textbf{Nivel 4:} & PV~+10 & PM~+5 & DEF~+1 & FUE~+1 \\ 
 \textbf{Nivel 5:} & PV~+10 & PM~+5 & FUE~+1 & DEF~+1 \\ 
 \textbf{Nivel 6:} & PV~+10 & PM~+10 & RES~+1 &        \\ 
 \textbf{Nivel 7:} & PV~+10 & PM~+5 & FUE~+1 & DEF~+1 \\
 \textbf{Nivel 8:} & PV~+10 & PM~+5 & RES~+1 & DEF~+1 \\
 \textbf{Nivel 9:} & PV~+10 & PM~+5 & DEF~+2 & 		\\ 
 \textbf{Nivel 10:}& PV~+10 & PM~+5 & FUE~+2 &        \\
} {Provocar} { Siempre que \hyperlink{action}{Ataques} a un enemigo con éxito, puedes intentar provocarlo. Si lo haces, el objetivo debe hacer una tirada con DC~7. Si falla, debe dirigir una acción hacia ti en su siguiente turno, si es posible. } {Bloqueo} { Cuando un enemigo que se encuentre a 1u de ti intenta alejarse de tu posición, debe hacer una tirada con DC 7. Si falla, sufre \hyperlink{status}{Inmóvil} hasta el comienzo de su próximo turno, evitando que se mueva más en este turno. }
\vfill
\atypet{Paladín} { \textbf{Level 4:} & PV~+10 & PM~+10 & DEF~+1 \\
 \textbf{Nivel 5:} & PV~+10 & PM~+5 & RES~+1 & DEF~+1 \\ 
 \textbf{Nivel 6:} & PV~+10 & PM~+5 & FUE~+2 &        \\
 \textbf{Nivel 7:} & PV~+10 & PM~+5 & RES +2 &        \\
 \textbf{Nivel 8:} & PV~+10 & PM~+5 & FUE~+1 & DEF~+1 \\
 \textbf{Nivel 9:} & PV~+10 & PM~+10 & FUE~+1 &        \\
 \textbf{Nivel 10:}& PV~+5 & PM~+10 & RES~+1 & DEF +1 \\
} {Guardia Sagrada} { Siempre que haya un aliado a 1u de ti, ambos reciben \hyperlink{status}{Reflejos}. Este beneficio no puede aplicarse a más de un aliado al mismo tiempo. } {Cubrir} { Siempre que un aliado que se encuentre a 1u de ti reciba daño \hyperlink{type}{Físico}, puedes elegir recibir la mitad del daño total en vez de que tu aliado reciba todo el daño. }
\pagebreak \\
\noindent {\Large\color{accent}\bf \uline{Habilidades\phantom{y}\hfill}}\\\\
\techt{Guardia}{3}{0t}{Único}{Tú}{ Obtienes \hyperlink{status}{aumDEF} hasta el final de tu próximo turno. }{\enndef}{1} \techt{Primeros Auxilios}{5}{0t}{Único}{1u}{ Elige a un objetivo que haya recibido daño este turno o el anterior (incluido tú). El objetivo recupera 2d de PV. }{}{2} \techt{Rompebrazo}{10}{0t}{Único}{Arma}{ Realiza un \hyperlink{action}{Ataque} sobre el objetivo. Si lo golpeas, el objetivo sufre \hyperlink{status}{disFUE} y \hyperlink{status}{disMAG} por 3 turnos además del daño infligido. }{\destr \demag}{3} \techt{Guardia Vital}{9}{1t}{Único}{Tú}{ Obtienes \hyperlink{status}{aumDEF} por 3 turnos. Cuando el efecto termine, recuperas 2d de PV. }{\enndef}{5} \spellt{Muro de Tierra}{10}{1t}{3u (línea)}{3u}{ Creas un muro de 3u de alto y ancho que bloquea el paso. El muro se rompe después de 5 turnos o tras sufrir un total de 30 puntos de daño. }{}{6} \techt{Hostigar}{8}{1t}{Único}{3u}{ El objetivo hace una tirada con DC 8. Si falla, queda \hyperlink{status}{Inmóvil} por 3 turnos. }{\immobile}{7} \spellt{Astra}{11}{1t}{Único}{3u}{ Por los próximos 3 turnos, el objetivo es \hyperlink{status}{Inmune} a todos los \hyperlink{status}{Estados Alterados}. }{}{8} \techt{Segar}{16}{0t}{Único}{Arma}{ Realiza un \hyperlink{action}{Ataque} sobre el objetivo. Si lo golpeas, el objetivo recibe como daño la diferencia entre tus PV actuales y tus PV máximos en lugar del daño normal. }{}{9} \techt{Guardia Total}{30}{1t}{Único}{Tú}{ Por los próximos 3 turnos, no recibes ningún daño. }{}{10}
\pagebreak
