\thispagestyle{empty}
\subsection*{\huge Tirador}
\vspace{0.3cm}
"Yo interpreto al líder, ¿quién más?" \\
\indent -- Balthier 
\vspace{0.3cm} \\
Los Tiradores son expertos en todo tipo de armas a distancia de gran alcance. Los Tiradores más habilidosos pueden ver a través de sus enemigos, lo que les permite conocer los puntos fuertes y débiles del objetivo. Por lo tanto, no solo pueden hacer daño significativo a distancia, sino también desactivar a los enemigos con técnicas especiales. Para el Tirador solo hay una regla: cazar o ser cazado.
\vfill
\battrt { \textbf{Nivel 1:} & PV~+19 & PM~+12 & AGI~+2 & FUE~+1 \\
 \textbf{Nivel 2:} & PV~+5 & PM~+10 & FUE~+1 & DEF~+1 \\
 \textbf{Nivel 3:} & PV~+10 & PM~+5 & FUE~+1 & RES~+1 \\
}{Arco, Arma de Fuego}{Armadura Ligera}
\vfill
\atypet{Explorador} { \textbf{Nivel 4:} & PV~+10 & PM~+10 & FUE~+1 &        \\
 \textbf{Nivel 5:} & PV~+5 & PM~+10 & DEF~+2 &        \\
 \textbf{Nivel 6:} & PV~+10 & PM~+10 & RES~+1 &        \\
 \textbf{Nivel 7:} & PV~+5 & PM~+10 & FUE~+1 & RES~+1 \\
 \textbf{Nivel 8:} & PV~+5 & PM~+5 & RES~+1 & DEF~+2 \\
 \textbf{Nivel 9:} & PV~+5 & PM~+10 & RES~+1 & FUE~+1 \\
 \textbf{Nivel 10:}& PV~+10 & PM~+5 & FUE~+2 &        \\
} {Retroceso} { Siempre que logres realizar un \hyperlink{action}{Ataque} con éxito, puedes moverte inmediatamente 1u (incluso cuando estés usando un arco). } {Evasión Mágica} { Puedes evadir \hyperlink{action}{Magia} si pasas una tirada de evasión de la misma manera en la que evades los \hyperlink{action}{Ataques}. }
\vfill
\atypet{Francotirador} { \textbf{Nivel 4:} & PV~+5 & PM~+5 & FUE~+2 & RES~+1 \\
 \textbf{Nivel 5:} & PV~+10 & PM~+10 & DEF~+1 &		  \\
 \textbf{Nivel 6:} & PV~+5 & PM~+10 & FUE~+1 & DEF~+1 \\
 \textbf{Nivel 7:} & PV~+5 & PM~+5 & FUE~+1 & RES~+2 \\
 \textbf{Nivel 8:} & PV~+5 & PM~+10 & DEF~+1 & RES~+1 \\
 \textbf{Nivel 9:} & PV~+10 & PM~+5 & FUE~+2 &        \\
 \textbf{Nivel 10:}& PV~+5 & PM~+5 & RES~+1 & FUE~+2 \\
} {Concentración} { Siempre que realices un \hyperlink{action}{Ataque} contra un enemigo, este tira con \hyperlink{check}{Desventaja} en su tirada de evasión. } {Objeto Automático} { Cuando sufras cualquier tipo de daño, puedes utilizar inmediatamente un \hyperlink{item}{Objeto}. Solo puedes utilizar este efecto una vez por turno. }
\pagebreak \\
\noindent {\Large\color{accent}\bf \uline{Habilidades\phantom{y}\hfill}}\\\\
 \techt{Gran Disparo}{3}{0t}{Único}{Arma} { Realiza un \hyperlink{action}{Ataque} sobre un objetivo. Si lo golpeas, el daño provocado ignora la DEF del objetivo. }{}{1} \techt{Colocar Trampa}{4}{1t}{1u}{Tú}{ Colocas una trampa en el lugar que te encuentras parado. Un enemigo que pasa por encima hace una tirada con DC 9. Si falla, recibe 2d de daño y queda \hyperlink{status}{Inmóvil} por 1 turno. La trampa desaparece una vez que fue activada. }{\immobile}{2} \spellt{Libra}{5}{0t}{Único}{3u} { Analizas con detalle al objetivo y descubres sus \hyperlink{type}{Resistencias}, \hyperlink{type}{Debilidades}, \hyperlink{type}{Inmunidades} y sus PV y PM actuales. }{}{3} \techt{Disparo Rápido}{9}{0t}{Único}{Arma} { Realizas un \hyperlink{action}{Ataque} e inmediatamente después puedes comenzar a utilizar una \hyperlink{action}{Habilidad} o un \hyperlink{item}{Objeto} en el mismo turno. }{}{5} \techt{Disparo Penetrante}{7}{0t}{10u (línea)}{Tú} { Realizas un \hyperlink{action}{Ataque} contra todos los objetivos en línea. Haz una sola tirada de daño que aplicará a todos los que fallen al intentar evadir. }{}{6} \techt{Munición Envenenada}{8}{0t}{Único}{Arma} { Realiza un \hyperlink{action}{Ataque} sobre un objetivo. Si lo golpeas, el daño infligido es \hyperlink{type}{Mágico} y el objetivo hace una tirada con DC 8. Si falla, queda \hyperlink{status}{Envenenado} por 3 turnos. }{\poison}{7} \techt{Apuntar a los PM}{4}{1t}{Único}{Arma} { Realiza un \hyperlink{action}{Ataque} sobre un objetivo. Si lo golpeas, reduce los PM del objetivo en vez de los PV con tu tirada de daño. }{}{8} \techt{Bomba de Humo}{12}{1t}{3u}{5u} { Crea una nube de humo que inhibe la visión dentro del área de efecto por 5 turnos. Todos los que estén en la nube sufren \hyperlink{status}{Ceguera}, pero obtienen \hyperlink{status}{Reflejos}. }{\blind\blink}{9} \techt{Ráfaga}{22}{1t}{Único}{Tú} { Haces 2 \hyperlink{action}{Ataques} consecutivos como acción en cada uno de tus turnos (el efecto dura 3 turnos). Siempre que este efecto esté activo, no podrás moverte ni realizar otras acciones. Puedes elegir finalizar el efecto al principio de cada turno. }{}{10}
\pagebreak
