\thispagestyle{empty}
\subsection*{\huge Mago Rojo}
\vspace{0.3cm}
"Te mostraré por qué mi nombre significa 'rayo'" \\
\indent -- Lightning 
\vspace{0.3cm} \\
Los Magos Rojos son seres muy versátiles que poseen una amplia variedad de habilidades. Son capaces de utilizar hechizos ofensivos y defensivos, pero también saben defenderse en el combate cuerpo a cuerpo. Aunque no destacan en ninguna disciplina, su increíble flexibilidad y velocidad hacen que los Magos Rojos sean una fuerza a tener en cuenta.
\vfill
\battrt { \textbf{Nivel 1:} & PV~+20 & PM~+16 & AGI~+3 & FUE~+1 \\
 \textbf{Nivel 2:} & PV~+5 & PM~+10 & MAG~+1 & DEF~+1 \\
 \textbf{Nivel 3:} & PV~+10 & PM~+5 & RES~+2 &        \\
}{Bastón, Espada}{Armadura Liviana, Túnica}
\vfill
\atypet{Fulminador} { \textbf{Nivel 4:} & PV~+5 & PM~+10 & MAG~+1 & RES~+1 \\ 
 \textbf{Nivel 5:} & PV~+10 & PM~+5 & FUE~+2 &        \\ 
 \textbf{Nivel 6:} & PV~+5 & PM~+10 & MAG~+2 &        \\ 
 \textbf{Nivel 7:} & PV~+10 & PM~+5 & FUE~+1 & DEF~+1 \\ 
 \textbf{Nivel 8:} & PV~+5 & PM~+10 & FUE~+1 & MAG~+1 \\ 
 \textbf{Nivel 9:} & PV~+10 & PM~+10 & RES~+1 &        \\ 
 \textbf{Nivel 10:}& PV~+5 & PM~+10 & MAG~+1 & DEF~+1 \\ 
} {Abrumar} { Siempre que inflijas daño a un enemigo que ya haya sufrido daño en el turno anterior, el daño que inflijas ignora la DEF y RES del objetivo. } {Lanzamiento Rápido} { Siempre que sufras daño mientras no estés concentrándote, puedes utilizar inmediatamente una habilidad sin tiempo de preparación. Solo puedes utilizar este efecto una vez por turno. }
\vfill
\atypet{Hoja Mágica} { \textbf{Nivel 4:} & PV~+10 & PM~+5 & FUE~+1 & DEF~+1 \\ 
 \textbf{Nivel 5:} & PV~+5 & PM~+10 & MAG~+2 & 		  \\ 
 \textbf{Nivel 6:} & PV~+10 & PM~+5 & FUE~+2 &		  \\ 
 \textbf{Nivel 7:} & PV~+5 & PM~+10 & DEF~+1 & FUE~+1 \\ 
 \textbf{Nivel 8:} & PV~+10 & PM~+5 & FUE~+1 & MAG~+1 \\ 
 \textbf{Nivel 9:} & PV~+5 & PM~+10 & DEF~+1 & RES~+1 \\ 
 \textbf{Nivel 10:}& PV~+10 & PM~+10 & FUE~+1 \\ 
} {Arma Mágica} { Cuando lances \hyperlink{action}{Magia}, puedes elegir almacenar el hechizo en tu arma. Si lo haces, el hechizo solo cuesta la mitad de PM de lo normal. El hechizo tiene efecto en el siguiente objetivo que golpees con un \hyperlink{action}{Ataque} además del daño habitual. No puedes almacenar más de un hechizo al mismo tiempo dentro de tu arma. } {Escudo de Maná} { Cuando se reduzcan tus PV, puedes elegir reducir en cambio tus PM por la misma cantidad. }
\pagebreak \\
\noindent {\Large\color{accent}\bf \uline{Habilidades\phantom{y}\hfill}}\\\\
\spellt{Cura}{4}{1t}{Único}{3u} { El objetivo recupera 2d de PV. }{}{1} \spellt{Veneno}{6}{1t}{Único}{3u} { El objetivo hace una tirada con DC 8. Si falla, sufre \hyperlink{status}{Veneno} durante 3 turnos. }{\poison}{2}\spellt{Piro+}{8}{1t}{Único}{4u}{  Infliges 4d de daño de \hyperlink{type}{Fuego} al objetivo. }{\fire}{3} \spellt{Hielo+}{8}{1t}{Único}{4u} { Infliges 4d de daño de \hyperlink{type}{Hielo} al objetivo. }{\ice}{3} \spellt{Trueno+}{8}{1t}{Único}{4u} { Infliges 4d de daño de \hyperlink{type}{Eléctrico} al objetivo. }{\lightning}{3} \spellt{Silencio}{6}{1t}{Único}{3u} { El objetivo hace una tirada con DC 8. Si falla, queda en \hyperlink{status}{Silencio} por 3 turnos }{\silence}{5} \spellt{Esuna}{6}{1t}{Único}{3u} { Elimina todos los \hyperlink{status}{Estados Alterados} del objetivo excepto \hyperlink{status}{KO}. }{}{6} \techt{Impacto Elemental}{4}{0t}{Único}{Arma} { Elige un \hyperlink{type}{elemento} (p. ej.,  \hyperlink{type}{Fuego}) y realiza un \hyperlink{action}{Ataque}. Si golpeas, el daño es del tipo \hyperlink{type}{mágico } con el elemento elegido. }{}{7} \spellt{Muro}{10}{1t}{Único}{3u} { El objetivo obtiene \hyperlink{status}{aumDEF} y \hyperlink{status}{aumRES} por 3 turnos. }{\enndef \enres}{8} \spellt{Aquiles}{10}{1t}{Único}{3u} { El objetivo sufre \hyperlink{status}{disDEF} y \hyperlink{status}{disRES} por 3 turnos. }{\dedef \deres}{8} \spellt{AntiElemento}{12}{1t}{Único}{5u} { Elije un \hyperlink{type}{elemento} (p. ej., \hyperlink{type}{Fuego}). El objetivo no sufre ningún daño del elemento elegido durante 3 turnos. }{}{9} \spellt{Magia Doble}{4}{0t}{Único}{Tú} { Comienzas a preparar o lanzar simultáneamente dos hechizos que elijas, pero debes gastar los PM necesarios para ambos }{}{10}
\pagebreak