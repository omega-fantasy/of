\thispagestyle{empty}
\subsection*{\huge Ladrón}
\vspace{0.3cm}
"¡PREFIERO el término 'cazador de tesoros'!" \\
\indent -- Locke 
\vspace{0.3cm} \\
Los Ladrones son luchadores cuerpo a cuerpo extremadamente ágiles, que pueden atravesar rápidamente el campo de batalla y son difíciles de golpear con ataques físicos. Se destacan en tomar "prestados" el dinero y los objetos de los enemigos y tienen un sexto sentido para los negocios que valen la pena. Hay que tener cuidado cuando tratas con un Ladrón, pues siempre tienen un truco más bajo la manga que jamás esperarías.
\vfill
\battrt { \textbf{Nivel 1:} & PV +20 & PM~+14 & AGI~+4 &        \\
 \textbf{Nivel 2:} & PV~+5 & PM~+5 & FUE~+2 & DEF~+1 \\
 \textbf{Nivel 3:} & PV~+10 & PM~+10 & RES~+1 &        \\
}{Daga}{Armadura Ligera}
\vfill
\atypet{Asesino} { \textbf{Nivel 4:} & PV~+10 & PM~+5 & FUE~+1 & DEF~+1 \\ 
 \textbf{Nivel 5:} & PV~+10 & PM~+10 & FUE~+1 &        \\ 
 \textbf{Nivel 6:} & PV~+5 & PM~+5 & DEF~+2 & RES~+1 \\ 
 \textbf{Nivel 7:} & PV~+5 & PM~+10 & FUE~+2 &        \\ 
 \textbf{Nivel 8:} & PV~+10 & PM~+5 & RES~+2 &        \\ 
 \textbf{Nivel 9:} & PV~+5 & PM~+10 & FUE~+2 &        \\ 
 \textbf{Nivel 10:}& PV~+10 & PM~+5 & DEF~+1 & FUE~+1 \\ 
} {Primer Golpe} { Obtienes el mayor resultado posible en tu tirada de iniciativa al inicio de cada batalla. } {Contraataque} { Siempre que un enemigo te golpee con éxito con un \hyperlink{action}{Ataque}, inmediatamente puedes hacer un \hyperlink{action}{Ataque} contra él. }
\vfill
\atypet{Cazador de Tesoros} { \textbf{Nivel 4:} & PV~+5 & PM~+10 & RES~+1 & DEF~+1 \\ 
 \textbf{Nivel 5:} & PV~+10 & PM~+5 & FUE~+1 & DEF~+1 \\ 
 \textbf{Nivel 6:} & PV~+5 & PM~+10 & FUE~+1 & RES~+1 \\ 
 \textbf{Nivel 7:} & PV~+10 & PM~+10 & DEF~+1 & 	      \\ 
 \textbf{Nivel 8:} & PV~+5 & PM~+10 & FUE~+1 & DEF~+1 \\ 
 \textbf{Nivel 9:} & PV~+5 & PM~+10 & RES~+2 &  	  \\ 
 \textbf{Nivel 10:}& PV~+10 & PM~+10 & FUE~+1 &		  \\ 
} {Gillonario} { Cuando tú los eliminas, los enemigos otorgan el doble de Gil de lo que suelen otorgar. } {Contrarrobo} { Siempre que logres evadir un \hyperlink{action}{Ataque} de un enemigo, puedes utilizar inmediatamente la habilidad "Robar Gil" o "Robar Objeto" sobre él sin ningún costo. }
\pagebreak \\
\noindent {\Large\color{accent}\bf \uline{Habilidades\phantom{y}\hfill}}\\\\
\techt{\hypertarget{tab}{Robar Gil}}{3}{0t}{Único}{Arma} { Haz una tirada con DC 7. Si tienes éxito, "tomas prestados" hasta 2d~x~10 Giles del objetivo. }{}{1} \techt{Huir}{5}{1t}{2u}{Tú} { Por 3 turnos, tú y todos tus aliados dentro del área de efecto pueden moverse el doble de la distancia normal cuando estén alejándose de sus enemigos. }{}{2} \techt{Robar Objeto}{5}{0t}{Único}{Arma} { Haz una tirada con DC 7. Si tienes éxito, "tomas prestado" un \hyperlink{item}{Objeto} del objetivo. Una tirada de 1d decidirá el objeto que recibirás: entre 1-3 será una \hyperlink{item}{Poción}, 4 será un \hyperlink{item}{Remedio}, 5 será un \hyperlink{item}{Éter} y 6 será un \hyperlink{item}{Ala de Fénix}. El objeto también puede ser determinado de cualquier otra forma por el DJ. }{}{3} \techt{Invisible}{10}{1t}{Único}{Arma} { Pasas a estar invisible por 1 minuto (6 turnos) o hasta que realices alguna acción. Mientras estés invisible, obtienes \hyperlink{status}{Reflejos} y tienes \hyperlink{check}{Ventaja} en todas las tiradas relacionadas con robar. Además, si logras golpear a un enemigo con un \hyperlink{action}{Ataque} mientras estés invisible, automáticamente haces \hyperlink{action}{Daño Crítico}. }{\blink}{5} \techt{Asaltar}{6}{0t}{Único}{Arma}{ Realiza un \hyperlink{action}{Ataque} contra el objetivo. Si golpeas, haces daño y robas 4d~x~10 Giles. }{}{6} \techt{Mano Veloz}{7}{0t}{Único}{1u}{ Puedes utilizar 2 \hyperlink{action}{Objetos} en el mismo turno. }{}{7} \techt{Arrojar}{4}{0t}{Único}{4u}{ Arrojas algún objeto de tu equipo e infliges 8d de daño si es un arma y 5d de daño si es otro tipo de objeto. Luego, haz una tirada con DC 8. Si fallas, el objeto es destruido. }{}{8} \techt{Sobornar}{5}{1t}{Único}{1u}{ Le pagas cierta cantidad de Giles al objetivo y haces una tirada con DC 13 menos 1 por cada 100 Giles que pagues. Si tienes éxito, el objetivo deja el campo de batalla. Puede que algunos enemigos sean \hyperlink{status}{Inmunes} a este efecto. }{}{9} \techt{Imitar}{?}{0t}{?}{?}{ Utilizas una habilidad que haya sido utilizada en el campo de batalla por un aliado o enemigo durante el turno anterior. Al hacerlo debes respetar las características de la habilidad imitada, como el coste de PM, tiempo de lanzamiento, objetivo y alcance. }{}{10}
\pagebreak
