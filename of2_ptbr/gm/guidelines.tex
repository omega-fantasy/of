\ofsection{Mestre de jogo}
%
\ofquote{"Difícil... Não nos culpe. Culpe a si mesmo ou a Deus."\\}{Delita}\\\\
%
\includegraphics[width=\columnwidth]{./art/images/ff4.jpg}
%
\vfill
%
O \accf{Mestre de jogo} cria o cenário da aventura e desempenha o papel de todos os personagens não jogadores.
Além disso, ele descreve o ambiente ao redor dos protagonistas e decide o resultado da maioria das ações ao aplicar as regras do jogo.
No entanto, diferente dos jogadores, o MJ não está restrito às regras e pode arbitrar como desejar, quando necessário.
Não há uma única maneira de ser um MJ bem sucedido, encorajamos que se adote um estilo que traga diversão a todos os envolvidos.
%
\vfill
%
Portanto, este capítulo não foca na apresentação de regras ou conselhos.
Ao invés disso, na coleção de \accf{suplementos} opcionais, que você pode tanto usar diretamente ou como referência para criar o seu próprio conteúdo.
Eles não somente dão uma noção sobre os vários aspectos do papel do mestre de jogo, mas também diferentes direcionamentos que você pode tomar como o MJ.
O módulo atual pode ser dividido em duas categorias: \accf{conteúdo preparado} e \accf{regras opcionais}.
Esse oferece blocos de criação de mundo, como aventuras, cenários e monstros independentes e expansíveis.
Enquanto aquele, apresenta exemplos de como customizar regras, ao alterar as já existentes ou adicionar outras novas, de acordo com suas preferências.
O conteúdo preparado combina mais com iniciantes, embora recomende-se considerar as regras opcionais uma vez que já se reuniu experiencia o suficiente.
Todos os suplementos disponíveis estão listados a seguir, juntos a uma curta sinopse.
%
\newpage
%
{\large\accf{Conteúdo preparado:}}
%
\vfill
%
\accf{Bestiário:} discute orientações para criação de monstros e encontros de combate.
Também inclui uma coleção de inimigos preparados para serem usados diretamente.
%
\vfill
%
\accf{Caos em Cornelia:} uma aventura curta na qual o grupo tem que salvar uma princesa raptada. 
Contém conteúdo diverso, incluindo combate, interpretação e exploração. Bastante recomendável para iniciantes!
%
\vfill
%
\accf{Tumba de Raithwall:} uma aventura curta na qual o grupo tem que recuperar um artefato de uma tumba perigosa.
Focada na exploração de um ambiente preenchido de armadilhas e adversários.
%
\vfill
%
\accf{Maria \& Draco:} uma aventura one-shot na qual o grupo tem que garantir o sucesso de uma apresentação de opera.
Encoraja uma narrativa leve com momentos de interpretação interessantes.
%
\vfill
%
\accf{Cerco a Dollet:} uma aventura one-shot na qual o grupo tem que passar em um teste a fim de se juntar a uma força mercenária de elite.
Encoraja uma narrativa cheia de ação com bastante combate.
%
\vfill
%
\accf{Gold Saucer:} um parque de diversões onde o grupo pode relaxar e ganhar prêmios raros.
Focado na recriação de jogos e competições do parque.
%
\vfill
%
\accf{Cenário de Ivalice:} um documento bastante detalhado que trás à vida o mundo de Ivalice, incluindo sua história e geografia.
Você pode criar várias aventuras neste mundo ou usá-lo como exemplo para criar um cenário detalhado.
%
{\large\ofpar\ofrow\accf{Regras opcionais:}}
%
\vfill
%
\accf{Regras adicionais:} pequenas alterações e o acréscimo de outras, que o ajudam a customizar seu estilo de jogo.
%
\vfill
%
\accf{Raças:} regras e exemplos de como incorporar diferentes raças humanoides ao seu mundo. 
Oferecendo opções de criação de personagens extras para os jogadores, mas que também podem ajudar o MJ a criar um mundo de jogo mais interessante.
%
\vfill
%
\accf{Chocobo:} regras para incorporar grandes pássaros chamados de Chocobos como membros efetivos do grupo. 
Os jogadores podem criar Chocobos, usá-las como montarias e lutar ao lado delas em combate.
%
\vfill
%
\accf{Tríade tripla:} regras para um jogo de cartas divertido, permitindo ao grupo coletar e jogar cartas.
Perfeito para quem está procurando uma atividade secundária rápida para o grupo.
%
\vfill
%
\accf{Blitzball:} regras para um esporte de time parecido com polo aquático.
Perfeito para quem está procurando uma atividade secundária mais elaborada para o grupo.
%
\clearpage